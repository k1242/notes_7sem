\textbf{Т13}. Пусть интервал движения автобусов $T = 25$ мин, время пути автобуса $t = 2$ мин, и время пути пешехода $\tau = 15$ мин. Выделим интервал от момента, когда первый автобус выехал ($t=0$), до момента когда доехал второй автобус ($t=T+t$). Пешеход пересечётся с автобусом только при времени выхода $\sub{t}{п} \geq T+t-\tau$, а значит
\begin{equation*}
	P = 1 - \frac{T+t-\tau}{T} = \frac{13}{25} = 0.52.
\end{equation*}



\textbf{Т14}. Пусть первая точка $x \in [0, 1]$ -- координата первой точки, $y \in [0, 1]$ -- координата второй точки (выбираем $y>x$). Стороны треугольника тогда должны быть фиксированы 
\begin{equation*}
	a = x, \hspace{5 mm} 
	b = y-x, \hspace{5 mm} 
	c = 1-y,
	\hspace{10 mm} 
	a+b\leq c, 
	\hspace{5 mm} 
	a+c \leq b,
	\hspace{5 mm} 
	b+c \leq a,
\end{equation*}
которые задают на вероятностном пространстве трегольничек в $1/8$ от квадрата, и в $1/3$ от вероятностного пространства (в силу выбора $y>x$), таким образом искомая вероятность равна $1/4$. 





\textbf{Т15 (задача Дюффона)}. Считая $l \leq L$, можем написать геометрическую вероятность в виде
\begin{equation*}
	P = \frac{1}{\pi L} \int_{0}^{\pi} \int_{0}^{l \sin \theta} \d x \d \theta = \frac{2}{\pi} \frac{l}{L},
\end{equation*}
где $l$ -- длина иглы, $L$ -- расстояние между полосками. 



\textbf{Т16}. Аналогично пусть $x \in [0,1]$ -- координата первой точки, $y \in [0,1]$-- координата второй точки. Вероятность третьей точке быть между ними равна $|x-y|$, тогда
\begin{equation*}
	P = \int_{0}^{1} \d x \int_{0}^{1} \d y \ |x-y| = 
	\int_{0}^{1} dx\, \left(
		\int_{0}^{x} (x-y)\d y + \int_{x}^{1} (y-x) \d y 
	\right) = \frac{1}{3},
\end{equation*}
что вполне соответсвует интуиции. 