\textbf{Т18}. Исходя из того, что сумма по столбцам должна быть единичной, можем составить матрицу
\begin{equation*}
	\Pi = \begin{pmatrix}
	    0.4 & 0.3 & 0.3 \\
	    0.5 & 0.6 & 0.5 \\
	    0.1 & 0.1 & 0.2 \\
	\end{pmatrix},
	\hspace{10 mm} 
	\vc{p} = \begin{pmatrix}
		0.2 \\ 0.3 \\ 0.5
	\end{pmatrix},
\end{equation*}
тогда $\Pi \vc{p} = \left(0.32,\, 0.53,\, 0.15\right)$. Стационарное состояение можем найти через собственный вектор
\begin{equation*}
	\Pi \vc{q} = \vc{q},
	\hspace{0.5cm} \Rightarrow \hspace{0.5cm}
	\vc{q} = \frac{1}{9}\left(3,\, 5,\, 1\right).
\end{equation*}


\textbf{Т19}. Для двухуровенвой системы с матрицей перехода
\begin{equation*}
	\Pi = \begin{pmatrix}
	    1-\alpha & \alpha  \\
	    \beta & 1-\beta  \\
	\end{pmatrix},
\end{equation*}
найдём собственные векторы и перейдём в их базис
\begin{equation*}
	\vc{q} \Pi = \vc{q},
	\hspace{0.5cm} \Rightarrow \hspace{0.5cm}	
	\vc{q}_1 = \left(1/2,\, 1/2\right), \ \ \lambda_1 = 1
	\hspace{5 mm} 
	\vc{q}_2 = \left(\alpha,\, -\beta\right), \ \ \lambda_2 = 1- \alpha -\beta.
\end{equation*}
Тогда можем найти степень матрицы
\begin{equation*}
	\Pi = S \begin{pmatrix}
	    \lambda_1 & 0  \\
	    0 & \lambda_2  \\
	\end{pmatrix} S^{-1},
	\hspace{5 mm} 
	\Pi^n = S \begin{pmatrix}
	    \lambda_1^n & 0  \\
	    0 & \lambda_2^n  \\
	\end{pmatrix} S^{-1} = 
	\frac{1}{\alpha+\beta} \begin{pmatrix}
	    \alpha \lambda_2^n + \beta & \alpha (1-\lambda_2^n)  \\
	    \beta(1-\lambda_2^n) & \alpha + \beta \lambda_2^n  \\
	\end{pmatrix},
\end{equation*}
и соответственно предельное распределение $\left(1/2,\, 1/2\right)$.
