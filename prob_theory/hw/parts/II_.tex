\textbf{Т10}. Для $f_\xi (x) = \frac{1}{2} e^{-|x|}$, найдём $h(t)$:
\begin{equation*}
	h(t) = \frac{1}{2}\left(
		\int_{0}^{\infty} e^{(it-1)x} \d x + \int_{-\infty}^{0} e^{(it+1)x} \d x
	\right) = \frac{1}{t^2+1}.
\end{equation*}


\textbf{Т11}. Для нормального распределения с параметрами $(a, \sigma^2)$, можем вспомнить
\begin{equation*}
	h_{a \xi  + b} = e^{i t b} h_{\xi} (at),
	\hspace{0.5cm} \Rightarrow \hspace{0.5cm}
	h(t) = e^{- \frac{1}{2}t^2 \sigma^2} e^{i t a}.
\end{equation*}


\textbf{Т12}. Обратная задача, найдем распределения через обратный Фурье образ $\cos t$:
\begin{equation*}
	{\mathcal F}^{-1}[\cos t] = \frac{1}{2}\left(
		\delta(x-1) + \delta(x+1)
	\right),
	\hspace{0.5cm} \Rightarrow \hspace{0.5cm}
	F(x) = \left\{\begin{aligned}
	    &0, &x \leq 1, \\
	    &1/2, &-1 < x \leq 1, \\
	    &1, &1 < x.
	\end{aligned}\right.
\end{equation*}
Теперь для $h_2(t)$:
\begin{equation*}
	{\mathcal F}^{-1}\left[\tfrac{1}{2} + \tfrac{1}{2} \cos t + \tfrac{i}{6} \sin t\right] = \frac{1}{3} \delta(x-1) + \frac{1}{2} \delta(x) + \frac{1}{6} \delta(x+1).
\end{equation*}


\textbf{Т13}. Найдём функцию распределения для $h(t) = e^{-t^2} \cos t$. В принципе можно снова вспомнить $h_{a \xi + b}$ и сразу написать ответ:
\begin{equation*}
	h(t) = \frac{1}{2}\left(
		e^{- \frac{1}{4}(x+1)^2} +  e^{- \frac{1}{4} (x-1)^2}
	\right) = \frac{e^{-1/4}}{2 \sqrt{\pi}} e^{-x^2/4} \ch\left(\frac{x}{2}\right).
\end{equation*}

\textbf{Т14}. Характеристическая функция должна быть равномерно непрерывной. Функция $\cos t^2$ не является равномерно непрерывной, в силу неограниченности производной $-2 \sin(t^2) t$. 

\textbf{Т15}. Вспомнив характеристическую функцию для распределения Пуассона
\begin{equation*}
	f_{\xi_\lambda} = \exp\left(
		\lambda(e^{it}-1)
	\right),
\end{equation*}
можем получить для случайной величина $\eta = \frac{1}{\sqrt{\lambda}}(\xi_\lambda - \lambda)$ характеристическую функцию
\begin{equation*}
	f_\eta = \exp\left(
		-it \sqrt{\lambda} + \lambda\left(e^{it/\sqrt{\lambda}}-1\right)
	\right) = \exp\left(
		-t^2 + o(1/\sqrt{\lambda})
	\right),
	\hspace{0.5cm} \Rightarrow \hspace{0.5cm}
	\lim_{\lambda \to \infty} P\left(
		\tfrac{\xi_\lambda-\lambda}{\sqrt{\lambda}} \leq x
	\right) = \int_{-\infty}^{x} e^{-y^2} \d y.
\end{equation*}


\textbf{Т16}. Теперь можем найти предел
\begin{equation*}
	\lim_{n \to \infty} e^{-n} \sum_{k=1}^{n} \frac{n^k}{k!} = \lim_{n \to \infty} P(\xi_n \leq n) = P\left(
		\tfrac{1}{\sqrt{n}}(\xi_n - n) \leq 0
	\right) = \Phi(0) = \frac{1}{2}.
\end{equation*}



