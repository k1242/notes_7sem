\textbf{Т17}. \red{\xmark}


\textbf{Т18}. \red{\xmark}


\textbf{Т19}. \red{\xmark}


\textbf{Т20}. \red{\xmark}



\textbf{Т21}. \red{\xmark}




\textbf{Т22}. Пусть $A$ -- потопление корабля, гипотезы $H_m$ -- попадание в корабль $m$ торпед, $m = 1,\,  \ldots,\, n$:
\begin{equation*}
	P(H_m) = C_n^m p^m (1-p)^{n-m}.
\end{equation*}
Найдём $P(A|H_m)$. Для $m=1$ по условию $P(A|H_1)=0$, для $m \geq 2$ корабль не потапили только если все торпеды в одном отсеке:
\begin{equation*}
	P(A|H_m) = 1 - k \left(\frac{1}{k}\right)^m = 1 - k^{1-m},
\end{equation*}
тогда полная вероятность потопления 
\begin{equation*}
	P(A) = \sum_{m=2}^{n} P(H_m) P(A|H_m) = \sum_{m=2}^{n} C_n^m p^m (1-p)^{n-m} \left(1-k^{1-m}\right).
\end{equation*}



\textbf{Т23}. \red{\xmark}