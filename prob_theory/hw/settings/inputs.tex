% document's head
% \phantom{42}

\begin{center}
    \LARGE \textsc{Первый этап отбора на пратику}
\end{center}

\begin{flushright}
	Примак Евгений Алексеевич \\
	2 курс ФОПФ
\end{flushright}

\hrule

\phantom{42}

\newpage


% \input{parts/test}

\section{Первое задание}

\subsection{Вероятностное пространство}

\subsection*{Свойства вероятности}



\textbf{Т1}. 
Найдём все события $X$: $\overline{XA} \cup \overline{X \bar{A}} = B$. Вспоминая $\overline{\cup A} = \cap \bar{A}$ и $\overline{\cap A} = \cup \bar{A}$, получаем
\begin{equation*}
	XX + (A+\bar{A})X + A \bar{A} = \bar{B},
	\hspace{0.5cm} \Rightarrow \hspace{0.5cm}	
	X = \bar{B}.
\end{equation*}


\textbf{Т2}.
Теперь $A X = AB$. Учитывая, что $A \bar{A} = \varnothing$, находим $X = B \cup \bar{A}$. 


\textbf{Т3}. Покажем, что $P\left(\bigcup_{k=1}^n  A_k\right) + P\left(\bigcap_{k=1}^n \bar{A}_k\right) = 1$, действительно:
\begin{equation*}
	P\left(\bigcup  A_k\right) + P\left(\overline{\bigcup A_k}\right) = 1,
	\hspace{0.5cm} \Leftarrow \hspace{0.5cm}
	P(X) + P(\bar{X}) = P(X \cup \bar{X}) = P(\Omega) = 1.
\end{equation*}


\textbf{Т4}. \red{\xmark}

\textbf{Т5}. \red{\xmark}


\textbf{Т6}. Покажем, что $\forall A, B$ $|P(AB)-P(A)P(B)|\leq \frac{1}{4}$. По отдельности, т.к. $P(A) \geq P(AB)$ и $P(B) \geq P(AB)$ находим
\begin{equation*}
	P(AB) - P(A)P(B) \leq P(AB) - P(AB)^2 = P(AB) (1-P(AB)) \leq \tfrac{1}{4}.
\end{equation*}
Теперь с другой стороны, вспоминая $P(AB) + P(A \bar{B}) = P(A)$,  находим
\begin{equation*}
	P(A) P(B) - P(AB) = P(A) P(B) - (P(A)-P(A \bar{B})) - P(A) P(\bar{B}) + P(A) P(\bar{B}) = P(A \bar{B}) - P(A) P(B) \leq \tfrac{1}{4},
\end{equation*}
в силу уже доказанного неравенства.

\textbf{Т7}. \red{\xmark}





\subsection*{Комбинаторика}
\textbf{Т8}. \red{\xmark}

\textbf{Т9}. \red{\xmark}

\textbf{Т10}. Найдём вероятность обнаружить всё четыре масти, взяв 6 карт из колоды. Всего карт каждой масти 13. Пусть событие
$A_i$ -- не вытащили карты $i$-й масти. Нам нужно найти $p = 1-P(A_1\cup A_2\cup A_3\cup A_4)$:
\begin{equation*}
	P(A_1\cup A_2\cup A_3\cup A_4\cup) = 4 P(A_i) - 6 P(A_i \cap A_j) + 4 P(A_i \cap A_j \cap A_k),
\end{equation*}
так как масти эквивалентны и $P(A_1\cap A_2\cap A_3\cap A_4\cap) = 0.$ Размер $|\Omega| = C^6_{52}$, тогда
\begin{equation*}
	P(A_i) = \frac{C^6_{39}}{C^6_{52}}, 
	\hspace{5 mm} 
	P(A_i \cap A_j) = \frac{C^6_{26}}{C^6_{52}},
	\hspace{5 mm} 
	P(A_i \cap A_j \cap A_k) = \frac{C^6_{13}}{C^6_{52}},
\end{equation*}
откуда находим
\begin{equation*}
	p = 1 - \frac{4 C^6_{39} - 6 C^6_{26} + 4 C^6_{13}}{C^6_{52}} \approx 0.43.
\end{equation*}


\textbf{Т11}. Пусть $A_i$ -- $i$-е письмо в нужном конверте, $A = A_1 \cup \ldots \cup A_n$. Тогда по формуле включений-исключений
\begin{equation*}
	P(A) = n P(A_1) - C_n^2 P(A_i A_j) + C_n^3 P(A_i A_j A_k) - \ldots,
\end{equation*}
где соответствующие вероятности 
\begin{equation*}
	P(A_i) = \frac{1}{n},
	\hspace{5 mm} 
	P(A_i A_j) = \frac{(n-2)!}{n!} = \frac{1}{n (n-1)},
	\hspace{5 mm} \ldots
\end{equation*}
Подставляя, находим
\begin{equation*}
	P(A) = 1 - \frac{1}{2!} + \frac{1}{3!} - \frac{1}{4!} + \ldots = \sum_{k=1}^{n} \frac{(-1)^{k+1}}{k!},
	\hspace{10 mm} 
	1-P(A) = \sum_{k=0}^{n} \frac{(-1)^k}{k!} \underset{n \to \infty}{\to} e^{-1}.
\end{equation*}


\textbf{Т12}. Раз половина людей имеет 50-рублевые купюры, а друга половина 100-рублевые, то можем сопоставить первым открывающие скобки, а вторым закрывающие. Таким образом задача сводится к подсчёту количества правильных скобочных последовательностей. Мощность $|\Omega| = C_{2n}^n$, тогда
\begin{equation*}
	P = \frac{C_n}{C_{2n^n}} = \frac{1}{n+1},
\end{equation*}
где числа Каталана $C_n = C_{2n}^n - C_{2n}^{n-1} = \frac{1}{n+1} C_{2n}^n$.






\subsection*{Геометрическая вероятность}
\textbf{Т13}. Пусть интервал движения автобусов $T = 25$ мин, время пути автобуса $t = 2$ мин, и время пути пешехода $\tau = 15$ мин. Выделим интервал от момента, когда первый автобус выехал ($t=0$), до момента когда доехал второй автобус ($t=T+t$). Пешеход пересечётся с автобусом только при времени выхода $\sub{t}{п} \geq T+t-\tau$, а значит
\begin{equation*}
	P = 1 - \frac{T+t-\tau}{T} = \frac{13}{25} = 0.52.
\end{equation*}



\textbf{Т14}. Пусть первая точка $x \in [0, 1]$ -- координата первой точки, $y \in [0, 1]$ -- координата второй точки (выбираем $y>x$). Стороны треугольника тогда должны быть фиксированы 
\begin{equation*}
	a = x, \hspace{5 mm} 
	b = y-x, \hspace{5 mm} 
	c = 1-y,
	\hspace{10 mm} 
	a+b\leq c, 
	\hspace{5 mm} 
	a+c \leq b,
	\hspace{5 mm} 
	b+c \leq a,
\end{equation*}
которые задают на вероятностном пространстве трегольничек в $1/8$ от квадрата, и в $1/3$ от вероятностного пространства (в силу выбора $y>x$), таким образом искомая вероятность равна $1/4$. 





\textbf{Т15 (задача Дюффона)}. Считая $l \leq L$, можем написать геометрическую вероятность в виде
\begin{equation*}
	P = \frac{1}{\pi L} \int_{0}^{\pi} \int_{0}^{l \sin \theta} \d x \d \theta = \frac{2}{\pi} \frac{l}{L},
\end{equation*}
где $l$ -- длина иглы, $L$ -- расстояние между полосками. 



\textbf{Т16}. Аналогично пусть $x \in [0,1]$ -- координата первой точки, $y \in [0,1]$-- координата второй точки. Вероятность третьей точке быть между ними равна $|x-y|$, тогда
\begin{equation*}
	P = \int_{0}^{1} \d x \int_{0}^{1} \d y \ |x-y| = 
	\int_{0}^{1} dx\, \left(
		\int_{0}^{x} (x-y)\d y + \int_{x}^{1} (y-x) \d y 
	\right) = \frac{1}{3},
\end{equation*}
что вполне соответсвует интуиции. 




\subsection{Условные вероятности и формула Байеса}
\textbf{Т17}. \red{\xmark}


\textbf{Т18}. \red{\xmark}


\textbf{Т19}. \red{\xmark}


\textbf{Т20}. \red{\xmark}



\textbf{Т21}. \red{\xmark}




\textbf{Т22}. Пусть $A$ -- потопление корабля, гипотезы $H_m$ -- попадание в корабль $m$ торпед, $m = 1,\,  \ldots,\, n$:
\begin{equation*}
	P(H_m) = C_n^m p^m (1-p)^{n-m}.
\end{equation*}
Найдём $P(A|H_m)$. Для $m=1$ по условию $P(A|H_1)=0$, для $m \geq 2$ корабль не потапили только если все торпеды в одном отсеке:
\begin{equation*}
	P(A|H_m) = 1 - k \left(\frac{1}{k}\right)^m = 1 - k^{1-m},
\end{equation*}
тогда полная вероятность потопления 
\begin{equation*}
	P(A) = \sum_{m=2}^{n} P(H_m) P(A|H_m) = \sum_{m=2}^{n} C_n^m p^m (1-p)^{n-m} \left(1-k^{1-m}\right).
\end{equation*}



\textbf{Т23}. \red{\xmark}


\subsection{Случайные величины}
\textbf{Т24}. \red{\xmark}


\textbf{Т25}. \red{\xmark}



\textbf{Т26}. В квадрат брошена точка в $(\xi_1,\, \xi_2)$. Найдём функцию распределения и плотность $\eta = \xi_1 + \xi_2$. Обозначив плотность распределения за $f_1$ и $f_2$ соответсвенно, можем найти
\begin{equation*}
	f_\eta (y) = \int_{0}^{1} f_1(y-x_2) f_2(x_2) \d x_2 = \int_{\max(0, y-1)}^{\min(1,y)} \d x_2 = \left\{\begin{aligned}
	    &y, &0 \leq y < 1, \\
	    &2-y, &1 \leq y \leq 2.
	\end{aligned}\right.
\end{equation*}
Интегрируя, находим $F_\eta (y)$:
\begin{equation*}
	F_\eta (y) = \left\{\begin{aligned}
	    &\tfrac{1}{2}y^2, & 0\leq y < 1, \\
	    &2y- \tfrac{1}{2}y^2 - 1, &1\leq y \leq 2.
	\end{aligned}\right.
\end{equation*}


\textbf{Т27}. Пусть $\xi_1,\, \xi_2 \in \text{P}_\lambda$ -- независимые случайные величины. Для суммы $\xi_1 + \xi_2 = \eta$, $x_1 + x_2 = y$можем найти 
\begin{equation*}
	P(\eta = y) = \sum_{x_1, x_2 \colon x_1 + x_2 = y} P_{\xi_1}(x_1) P_{\xi_2}(x_2) = \sum_{0 \leq x_1 \leq y}  \frac{\lambda^y e^{-2\lambda}}{x_1! (y-x_1)!} = \frac{\lambda^y e^{-2\lambda}}{y!} 2^y.
\end{equation*}
Условное распределение $\xi_1$ при известной $\eta = \xi_1 + \xi_2$ найдём в виде
\begin{equation*}
	\frac{P(\xi_1 = x,\, \eta=y)}{P(y=\eta)} = \frac{P(\xi_1=x) P(\xi_2=y-x)}{P(y=\eta)} =  \frac{C_y^x}{2^y}.
\end{equation*}


\textbf{Т28}. Известно, что случайная величина $\xi$ имеет строго возрастающую непрерывную функцию распределения $F_\xi (x)$. Найдём распределение случайное величины $\eta = F_\xi(\xi)$:
\begin{equation*}
	F_\eta (x) = P(F(\xi) < x) = \left\{\begin{aligned}
	    &0, &x \leq 0,\\
	    &P(\xi<F^{-1}(x)), &x \in (0, 1),\\
	    &1, &x \geq 1.\\
	\end{aligned}\right.
\end{equation*}
По определению $P(\xi < F^{-1} (x)) = F(F^{-1}(x)) = x$, получается $\eta \in U_{0,1}$. 


\textbf{Т29}. Пусть $\xi \in N_{0, 1}$, найдём функцию распределения $F_{\xi^2}(x)$ и плотность $f_{\xi^2}(x)$. Нас  интересуют все такие $x^2 = y$ или $x = \sqrt{y}$, тогда
\begin{equation*}
	f_\xi (x) \d x \ \overset{x = \sqrt{y}}{\to} \ \frac{f_\xi(\sqrt{y})}{\sqrt{y}} \d y,
	\hspace{0.5cm} \Rightarrow \hspace{0.5cm}
	f_{\xi^2}(y) = \frac{1}{\sqrt{2\pi y}} e^{-y/2}.
\end{equation*}
Теперь по определению находим функцию распределения
\begin{equation*}
	F_{\xi^2}(z) = \int_{0}^{z} \frac{1}{\sqrt{2\pi}} \frac{e^{-y/2}}{\sqrt{y}} \d y = \frac{2}{\sqrt{2\pi}} \int_{0}^{z} e^{-\sqrt{y}^2/2} \d \sqrt{y} = \Erf\left(\sqrt{\tfrac{z}{2}}\right).
\end{equation*}


\textbf{Т30}. \red{\xmark}


\textbf{Т31}. Можем решать эту задачу в терминах геометрической вероятности, тогда на кубе $[0,1]^3$ задана система неравенств
\begin{equation*}
	|x-y| \geq \frac{1}{4},
	\hspace{5 mm} 
	|z-y| \geq \frac{1}{4},
	\hspace{5 mm} 
	|x-z| \geq \frac{1}{4},
\end{equation*}
которые формируют в кубе 6 одинаковых пирамидок объёма $1/48$, тогда искомая вероятность равна $1/8$.


\subsection{Математическое ожидание, дисперсия и ковариация}
\textbf{Т32}. \red{\xmark}


\textbf{Т33}. \red{\xmark}


\textbf{Т34}. \red{\xmark}


\textbf{Т34}. \red{\xmark}



\subsection{Математическое ожидание, дисперсия и ковариация}
\textbf{Т32}. \red{\xmark}


\textbf{Т33}. \red{\xmark}


\textbf{Т34}. \red{\xmark}


\textbf{Т34}. \red{\xmark}



\section{Второе задание}

\setcounter{subsection}{1}

\subsection{Метод характеристических функций}
\textbf{Т10}. Для $f_\xi (x) = \frac{1}{2} e^{-|x|}$, найдём $h(t)$:
\begin{equation*}
	h(t) = \frac{1}{2}\left(
		\int_{0}^{\infty} e^{(it-1)x} \d x + \int_{-\infty}^{0} e^{(it+1)x} \d x
	\right) = \frac{1}{t^2+1}.
\end{equation*}


\textbf{Т11}. Для нормального распределения с параметрами $(a, \sigma^2)$, можем вспомнить
\begin{equation*}
	h_{a \xi  + b} = e^{i t b} h_{\xi} (at),
	\hspace{0.5cm} \Rightarrow \hspace{0.5cm}
	h(t) = e^{- \frac{1}{2}t^2 \sigma^2} e^{i t a}.
\end{equation*}


\textbf{Т12}. Обратная задача, найдем распределения через обратный Фурье образ $\cos t$:
\begin{equation*}
	{\mathcal F}^{-1}[\cos t] = \frac{1}{2}\left(
		\delta(x-1) + \delta(x+1)
	\right),
	\hspace{0.5cm} \Rightarrow \hspace{0.5cm}
	F(x) = \left\{\begin{aligned}
	    &0, &x \leq 1, \\
	    &1/2, &-1 < x \leq 1, \\
	    &1, &1 < x.
	\end{aligned}\right.
\end{equation*}
Теперь для $h_2(t)$:
\begin{equation*}
	{\mathcal F}^{-1}\left[\tfrac{1}{2} + \tfrac{1}{2} \cos t + \tfrac{i}{6} \sin t\right] = \frac{1}{3} \delta(x-1) + \frac{1}{2} \delta(x) + \frac{1}{6} \delta(x+1).
\end{equation*}


\textbf{Т13}. Найдём функцию распределения для $h(t) = e^{-t^2} \cos t$. В принципе можно снова вспомнить $h_{a \xi + b}$ и сразу написать ответ:
\begin{equation*}
	h(t) = \frac{1}{2}\left(
		e^{- \frac{1}{4}(x+1)^2} +  e^{- \frac{1}{4} (x-1)^2}
	\right) = \frac{e^{-1/4}}{2 \sqrt{\pi}} e^{-x^2/4} \ch\left(\frac{x}{2}\right).
\end{equation*}

\textbf{Т14}. Характеристическая функция должна быть равномерно непрерывной. Функция $\cos t^2$ не является равномерно непрерывной, в силу неограниченности производной $-2 \sin(t^2) t$. 

\textbf{Т15}. Вспомнив характеристическую функцию для распределения Пуассона
\begin{equation*}
	f_{\xi_\lambda} = \exp\left(
		\lambda(e^{it}-1)
	\right),
\end{equation*}
можем получить для случайной величина $\eta = \frac{1}{\sqrt{\lambda}}(\xi_\lambda - \lambda)$ характеристическую функцию
\begin{equation*}
	f_\eta = \exp\left(
		-it \sqrt{\lambda} + \lambda\left(e^{it/\sqrt{\lambda}}-1\right)
	\right) = \exp\left(
		-t^2 + o(1/\sqrt{\lambda})
	\right),
	\hspace{0.5cm} \Rightarrow \hspace{0.5cm}
	\lim_{\lambda \to \infty} P\left(
		\tfrac{\xi_\lambda-\lambda}{\sqrt{\lambda}} \leq x
	\right) = \int_{-\infty}^{x} e^{-y^2} \d y.
\end{equation*}


\textbf{Т16}. Теперь можем найти предел
\begin{equation*}
	\lim_{n \to \infty} e^{-n} \sum_{k=1}^{n} \frac{n^k}{k!} = \lim_{n \to \infty} P(\xi_n \leq n) = P\left(
		\tfrac{1}{\sqrt{n}}(\xi_n - n) \leq 0
	\right) = \Phi(0) = \frac{1}{2}.
\end{equation*}





\subsection{Элементы теории случайных процессов}
\textbf{Т18}. Исходя из того, что сумма по столбцам должна быть единичной, можем составить матрицу
\begin{equation*}
	\Pi = \begin{pmatrix}
	    0.4 & 0.3 & 0.3 \\
	    0.5 & 0.6 & 0.5 \\
	    0.1 & 0.1 & 0.2 \\
	\end{pmatrix},
	\hspace{10 mm} 
	\vc{p} = \begin{pmatrix}
		0.2 \\ 0.3 \\ 0.5
	\end{pmatrix},
\end{equation*}
тогда $\Pi \vc{p} = \left(0.32,\, 0.53,\, 0.15\right)$. Стационарное состояение можем найти через собственный вектор
\begin{equation*}
	\Pi \vc{q} = \vc{q},
	\hspace{0.5cm} \Rightarrow \hspace{0.5cm}
	\vc{q} = \frac{1}{9}\left(3,\, 5,\, 1\right).
\end{equation*}


\textbf{Т19}. Для двухуровенвой системы с матрицей перехода
\begin{equation*}
	\Pi = \begin{pmatrix}
	    1-\alpha & \alpha  \\
	    \beta & 1-\beta  \\
	\end{pmatrix},
\end{equation*}
найдём собственные векторы и перейдём в их базис
\begin{equation*}
	\vc{q} \Pi = \vc{q},
	\hspace{0.5cm} \Rightarrow \hspace{0.5cm}	
	\vc{q}_1 = \left(1/2,\, 1/2\right), \ \ \lambda_1 = 1
	\hspace{5 mm} 
	\vc{q}_2 = \left(\alpha,\, -\beta\right), \ \ \lambda_2 = 1- \alpha -\beta.
\end{equation*}
Тогда можем найти степень матрицы
\begin{equation*}
	\Pi = S \begin{pmatrix}
	    \lambda_1 & 0  \\
	    0 & \lambda_2  \\
	\end{pmatrix} S^{-1},
	\hspace{5 mm} 
	\Pi^n = S \begin{pmatrix}
	    \lambda_1^n & 0  \\
	    0 & \lambda_2^n  \\
	\end{pmatrix} S^{-1} = 
	\frac{1}{\alpha+\beta} \begin{pmatrix}
	    \alpha \lambda_2^n + \beta & \alpha (1-\lambda_2^n)  \\
	    \beta(1-\lambda_2^n) & \alpha + \beta \lambda_2^n  \\
	\end{pmatrix},
\end{equation*}
и соответственно предельное распределение $\left(1/2,\, 1/2\right)$.





% \newpage
% \begin{to_thr}[о монотонной сходиомости]
    Пусть $\eta,\, \xi,\, \xi_1,\, \xi_2,\, \ldots$ -- случайные величины. Если $\xi_n \geq \eta$ для всех $n \geq 1$, $\E \eta > -\infty$ и $\xi_n \uparrow \xi$, то $\E \xi_n \uparrow \E \xi$. Если $\xi_n \leq \eta$ для всех $n \geq 1$, $\E \eta < \infty$ и $\xi_n \downarrow \xi$, то $\E \xi_n \downarrow \E \xi$. 
\end{to_thr}

% x_n \uparrow  x -- монотонно сходится  к x

\begin{to_thr}[лемма Фату]
    Пусть $\eta,\, \xi_1,\, \xi_2,\, \ldots$ -- случайные величины. Если $\xi_n \geq \eta$ для всех $n \geq 1$ и $\E \eta >  - \infty$, то $\E \underline{\lim} \xi_n \leq \underline{\lim} \E \xi_n$. Если $\xi_n \leq \eta$ для всех $n \geq 1$ и $\E \eta < \infty$, то $\overline{\lim} \E \xi_n \leq \E \overline{\lim} \xi_n$. Если $|\xi_n| \leq \eta$ для всех $n \geq 1$ и $\E \eta < \infty$, то $\E \underline{\lim} \xi_n \leq \underline{\lim} \E \xi_n \leq \overline{\lim} \E \xi_n \leq \E \overline{\lim} \xi_n$.
\end{to_thr}

% п.н. -- почти наверное

\begin{to_thr}[теорема Лебега о мажорируруемой сходимости]
    Пусть случайные величины таковы, что $|\xi_n| \leq \eta$, $\E \eta < \infty$ и $\xi_n \to \xi$ (п. н.). Тогда $\E |\xi| < \infty$ и $\E \xi_n \to \E \xi$, $\E |\xi_n-\xi| \to 0$. 
\end{to_thr}

\begin{to_con}
    Пусть $\eta,\, \xi,\, \xi_1,\, \xi_2,\, \ldots$ -- случайные величины такие, что $|\xi_n| \leq \eta$, $\xi_n \to \xi$ (п. н.) и $\E \eta^p < \infty$ для некоторого $p > 0$. Тогда $\E |\xi|^p < \infty$ и $\E |\xi-\xi_n|^p \to 0$. 
\end{to_con}

% характеристические функции
% сходимости, табличка















% \begin{gather*}
% 	I_{1.1} = \int \frac{\d x}{1+x},
% 	\hspace{5 mm} 
% 	I_{1.2} = \int \frac{\d x}{(1+x)(2+x)}, 
% 	\hspace{5 mm} 
% 	I_{1.3} = \int \frac{\d x}{1+x^2}; 
% 	\\
% 	I_{2.1} = \int_{-\infty}^{\infty} e^{-x^2} \d x,
% 	\hspace{5 mm} 
% 	I_{2.2} = \int_{-\infty}^{\infty} x^2 e^{-x^2} \d x, 
% 	\\
% 	I_{3.1} = \int_{0}^{\infty} x^n e^{-x} \d x;
% 	\\
% 	I_{4.1} = \int_{0}^{\pi/2} \sin^3(x) \d x,
% 	\hspace{5 mm} 
% 	I_{4.2} = \int_{0}^{\pi/2} \sin^n(x) \cos^m (x) \d x.
% \end{gather*}