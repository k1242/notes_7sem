\subsection*{Модуль 1}
\noindent Буду писать как-то очень кратко.
\begin{itemize}
	\item Покупательская способность --- возможность конвертировать деньги в определенный товар;
	\item Альтернативные издержки --- упущенная выгода наиболее прибыльного из невыбранных вариантов инвестирования;
	\item Экономика --- рациональное и эффективное вложение имеющихся ресурсов.
\end{itemize}	

\noindent Рассмотрим две ключевые функции банка.
\begin{itemize}
	\item Банковская процентная ставка --- планка экономики, по которой меряют эффективность инвестиционных проектов;
	\item Принятие на себя вкладов от населения и выдача кредитов предприятиям.
\end{itemize}

\noindent Существующие \textbf{типа вкладов}
\begin{itemize}
	\item Срочные;
	\item Бессрочные;
	\item До востребования;
	\item Долгосрочные.
\end{itemize}

\noindent \textbf{Типы условий}
\begin{itemize}
	\item Простая процентная ставка --- ставка начисляется на изначальное тело кредита;
	\item Сложная процентная ставка --- начисление на тело + процентные ставки полученные в конце года.
\end{itemize}

\noindent \textbf{Типы капитализаций} (начисление ставок за \textit{time})
\begin{itemize}
	\item Ежемесячная;
	\item Годовая;
	\item Ежедневная.
\end{itemize}

\noindent Сейчас будут достаточно умозрительные формулы, но тем не менее. И так, для \textbf{простой процентной ставки}
\begin{equation*}
	S_n = S_0(1+n \cdot r),
\end{equation*}
где $r$ -- процентная ставка, $n$ -- число временных промежутков (e.g. число лет), $S_0$ -- начальные инвестиции.

\noindent Для \textbf{сложной процентной ставки}
\begin{equation*}
	S_n = S_0 (1 + r)^n.
\end{equation*}
Не менее занятный факт, если ставка указана годовая, а капитализация ежемесячная, то каждый месяц\footnote{оставим это читателям в качестве упражнения.} начисляется... В задачах на такую тему нужно только следить простая или сложная ставка по месяцам и в году. Рассмотрим \textbf{общую формулу для сложной ставки}
\begin{equation*}
	j_{\text{ефф}} = \left(1 + \frac{r}{m}\right)^m - 1,
\end{equation*}
где $j_{\text{ефф}}$ -- процентная ставка, $m$ -- чило подпромежутков через которые начисляется сложная ставка (e.g. месяцев в году, кварталов в году).