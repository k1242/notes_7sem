\subsection*{продолжаем}
\begin{equation*}
	Z_{1 \text{ вращ}}^\text{орто} = \sum_{L = 1,2,5 \ldots} =3 e^{- 2 T_c/T} + 7 e^{-12 T_c/T} + \ldots
\end{equation*}
Энергия вращаетльного движения тогда
\begin{equation*}
	E_{1 \text{ вращ}}^\text{орто} = T^2 \frac{\partial}{\partial T} \ln Z_{1 \text{ вращ}}^\text{орто} = T^2 \frac{\partial}{\partial T} \ln 3 e^{- 2 T_c /T}\left(1 + \frac{7}{3} e^{-10 T_c/T} + \ldots \right)
	= T^2 \frac{\partial}{\partial T} \left(- \frac{2 T_c}{T} + \frac{7}{3} e^{-10 \frac{T_c}{T}}\right) = 2 T_c + \frac{70}{3} e^{- 10 T_c/T}.
\end{equation*}
И тогда
\begin{equation*}
	C_{1 \text{ вращ}}^\text{орто} = \frac{700}{3} \frac{T_c^2}{T^2} e^{-10 T_c/T} + \ldots.
\end{equation*}
\red{нарисовать график}

Может показаться, что к единице ничего не стремится, но
\begin{equation*}
	Z_{1 \text{ вращ}}^\text{орто} = \int_{0}^{+\infty} d L (2 L + 1) e^{-L(L+1)T_c/T} = \frac{T}{T_c}
	\hspace{0.5 cm}
	\leadsto
	\hspace{0.5 cm}
	E_{1 \text{ вращ}}^\text{орто} = T^2 \frac{\partial}{\partial T} \ln \frac{T}{T_c} = T
	\hspace{0.5 cm}
	\leadsto
	\hspace{0.5 cm}
	C_{1 \text{ вращ}}^\text{орто} = 1.
\end{equation*}
Теперь в качестве упражнения можно рассмотреть смесь в состоянии термодинамического равновесия орто + пара водород $N_\text{орто} + N_\text{пара} = N$. Показать, что
\begin{equation*}
	\frac{N_\text{орто}(T)}{N_\text{пара}(T)} = 3 \frac{Z_\text{орто}(T)}{Z_\text{пара}(T)}.
\end{equation*}
Перейдём к теории флуктуаций\footnote{Сейчас у меня сядет ноут, потому что в 520ГК не работает розетка. Поэтому возможно когда-то (когда ещё и первый семинар) появится и эта часть конспекта.}
\subsection*{Термодинамические квазиравновесные флуктуации}
\subsubsection*{Изолированное тело}
Рассматриваем изолированное тело, хотим посмотреть вероятность находится в неравновесном состоянии, задаваемом параметром $x$ -- отклонение от равновесного состояния. 
\begin{equation*}
	w(x) \sim \frac{\text{число всех микростостояний отвечающих } x}{\text{полное число всех возможных микросостояний}} \sim \text{статвес состояния } x \sim e^{S(x)}.
\end{equation*}
\begin{equation*}
	S(x) = S(x_0) + \Delta S,
	\hspace{1 cm}
	\Rightarrow
	\hspace{1 cm}
	w(x) \sim e^{\Delta S(x)}
\end{equation*}
где $S(x_0)$ -- отвечает равновесию системы.

\subsubsection*{Тело в среде}
Среда есть термостат. К телу относятся переменные $\{S,V,T,P\}$, а к термостату относятся $\{S_0, V_0, T_0, P_0\}$ -- и он всегда в равновесии.
Вводим снова параметр $x$ -- харакатеризующий отклонение состояние тела от равновесия
\begin{equation*}
	w(x) \sim e^{\Delta S_\text{tot}(x)},
	\hspace{1 cm}
	\Delta S_\text{tot} = \Delta S + \Delta S_0.
\end{equation*}
Мы считаем, что состояние тремостата остаётся равновесным. ТО есть $T_0$ и $P_0$ остаются постоянными.
Тогда для него справедлив второй закон термодинамики
\begin{equation*}
	\Delta S_0 = \frac{\delta Q_0}{T_0} = - \frac{\delta Q}{T_0}.
\end{equation*}
Ну а тепло ему, кроме как от тела получат неоткуда.

А вот тело, хоть и отклоняется от равновесия, но всегда можем записать первый закон
\begin{equation*}
	\delta Q = \Delta E + A_\text{против сил внешнего давления} = \Delta E + P_0 \Delta V,
\end{equation*}
так как внешнее давление это и есть $P_0$.
Таким образом получаем для энтропии теромтстата
\begin{equation*}
	\Delta S_0 = - \frac{\Delta E + P_0  \Delta V}{T_0}
	\hspace{1 cm}
	\Rightarrow
	\hspace{1 cm}
	\Delta S_\text{tot} = \Delta S - \frac{\Delta E + P_0 \Delta V}{T_0} = \frac{T_0 \Delta S - \Delta E - P_0  \Delta V}{T_0}.
\end{equation*}
Таким образом 
\begin{equation*}
	w(x) \sim \exp\left(\frac{T_0 \Delta S - \Delta E - P_0  \Delta V}{T_0}\right),
	\hspace{1 cm}
	x = \{\Delta S, \Delta V, \Delta E\}.
\end{equation*}

Теперь предположим, что флуктуации малы. Тогда $E(S,V)$ имеет тот же функциональный вид, тот же самый, как и в равновесии.
