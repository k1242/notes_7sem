\subsection*{Классическая система}
Классической системой будем называть случай, когда число мест, куда можно положить частицу много больше числа этех частиц $g_j \gg n_j$. Так называемый автобус поздней ночью.
Число различных квантовых состояний на уровне $\varepsilon_j$
\begin{equation*}
	W_j = \underbrace{g_j g_j \ldots g_j g_j}_{n_j} \frac{1}{n_j!} = \frac{g_j^{n_j}}{n_j!}.
\end{equation*}
Тогда число состояний системы
\begin{equation*}
	W = \prod_j W_j  = \prod_j \frac{g_j^{n_g}}{n_j !}.
\end{equation*}
Энтропия же в таком случае
\begin{equation*}
	S(\{n_j\}) = \ln W_j = \sum_j \ln \frac{g_j^{n_g}}{n_j !} = \sum_j \ln \left(\frac{e g_j}{n_j}\right)^{n_j}
	= \sum_j n_j \ln \frac{e g_j}{n_j},
\end{equation*}
где мы воспользовались формулой Стирлинга $n_j! \approx \left(\frac{n_j}{e}\right)^{n_j}$.

Ищем максимум $S(\{n)j\})$ при условиях, что $\sum_j n_j = N$ и $\sum_j n_j \varepsilon_j = E$.
По метода Лагранжа составляем
\begin{equation*}
	\tilde{S} = S + \alpha N + \beta E
	\hspace{1 cm}
	\leadsto
	\hspace{1 cm}
	\frac{\partial\tilde{S}}{\partial n_j} = 0,
\end{equation*}
ищем вот так вот набор, доставляющий ей максимум. Получаем его
\begin{equation*}
	n_i = g_i e^{-\mu/T} e^{- \varepsilon_i/T}.
\end{equation*}
Подставляя такой набор в $S(\{n_j\}) \longrightarrow S_\text{равн}$ получаем термодинамическую энтропию.

Посчитаем эту самую энтропию в максимуме
\begin{equation*}
	S_\text{max} = \sum_i n_i \ln \left(e\frac{1}{e^{\frac{\mu}{T} - \frac{e_i}{T}}}\right)
	= \sum_i n_i \left(1 - \frac{\mu}{T} + \frac{\varepsilon_i}{T}\right) 
	=- N - \frac{\mu}{T} N + \frac{1}{T} E.
\end{equation*}
Домножим на $T$
\begin{equation*}
	T S = T N - \mu N + E
	\hspace{1 cm}
	\Rightarrow
	\hspace{1 cm}
	\underbrace{\mu N}_{\Phi} = \underbrace{E - TS}_{F} + T N
	\hspace{1 cm}
	\Rightarrow
	\hspace{1 cm}
	\Phi = F + TN.
\end{equation*}
А про эти термодинамические потенциалы мы помним, что 
\begin{equation*}
	\Phi = F + P V
	\hspace{1.5 cm}
	\Rightarrow
	\hspace{1.5 cm}
	PV = NT.
\end{equation*}
Получили Уравнение Состояния Идеального Газа!

\subsection*{Большой Канонический Ансамбль}
Всё отличие новой задачи в том, что у нас есть выпрыгивания частиц из объёма $V$. Термостат нам задаёт $T$ и $\mu$, но сам по себе обменивается с телом порциями $\delta N$ и $\delta Q$.
Макрососотяние характеризуется набором $(T, V, \mu)$. Вероятность системы иметь $N$ частиц в квантовом состоянии $n$ задается выражением
\begin{equation*}
	W_{N,n} = \frac{1}{\mathcal{Z}} e^{\frac{N \mu - E_{N,n}}{T}}.
\end{equation*}
Это называется \textbf{большим каноническим распределением}, или \textbf{распределением Гиббса}.

Здесь $E_{N,n}$ -- уровни энергии системы, $\mathcal{Z} = \sum_{N,n} e^{(N \mu - E_{N,n})/T}$ -- \textbf{большая статистическая сумма}. При чём суммирование происходит по $N \in [0, \infty]$ и по $n$ пробегающем все квантовые состояния при заданном числе частиц $N$.

Так оказывается, что
\begin{equation*}
	\Omega = - T \ln \mathcal{Z}.
\end{equation*}
И в книжке мы можем найти убедительные доводы, а здесь я только напишу определение большого омега потенциала
\begin{equation*}
	\Omega(T,V,\mu) = F - \mu N.
\end{equation*}
Теперь можно вспомнить, что $F(T,V,N)$ следовательно
\begin{equation*}
	dF = - S d T - P d V + \mu d N
	\hspace{1 cm}
	\Rightarrow
	\hspace{1 cm}
	d \Omega = d F - d(\mu N) = - S d T - P d V - N d \mu.
\end{equation*}
Отсюда видно, что большой омега потенциал есть функция $\Omega(T,V,\mu)$, при чем $N = - \left(\frac{\partial \Omega}{\partial \mu}\right)_{T,V}$ -- термодинамическое число частиц. То есть $N_{\text{TD}} = \left\langle N_\text{факт}\right\rangle$.

Теперь научимся всем этим пользоваться\footnote{Не будем в дальнейшем путать все наши выкладки с обычной статсуммой $Z = \sum_n e^{-E_n/T}$. Тут $N$ фиксировано.}.

\textbf{Во-первых},  найдём $\langle N\rangle$ в системе (сама по себе она флуктуирующая величина).
\begin{equation*}
\begin{aligned}
	\langle N\rangle &= \sum_{N=0}^\infty \sum_n N W_{N,n} = \sum_{N,n} N \frac{1}{\mathcal{Z}} e^{\frac{N \mu - E_{N,n}}{T}} 
	= \sum_{N,n} \frac{1}{\mathcal{Z}} T \frac{\partial}{\partial \mu} e^{\frac{N \mu - E_{N,n}}{T}}
	=  \frac{1}{\mathcal{Z}} T \frac{\partial}{\partial \mu} \underbrace{\sum_{N,n}  e^{\frac{N \mu - E_{N,n}}{T}}}_\mathcal{Z}\\
	&= T \frac{1}{\mathcal{Z}} \frac{\partial}{\partial \mu} \mathcal{Z} = T \frac{\partial}{\partial \mu}\ln \mathcal{Z} = - \frac{\partial}{\partial\mu} \left(- T \ln \mathcal{Z}\right) = -\frac{\partial \Omega}{\partial\mu} = N_\text{TD}.
\end{aligned}
\end{equation*}
Мы обещали, что так будет? Получили!
Ещё наблюдая за всем этим действием мы замечаем, что операция $T \frac{\partial}{\partial\mu}$ -- очень полезная.

\textbf{Во-вторых}, давайте применим эту полезную операцию к тому же ещё раз!
\begin{equation*}
\begin{aligned}
	T \frac{\partial}{\partial\mu} \langle N\rangle
	&=
	T \frac{\partial}{\partial\mu} \sum_{N,n} \frac{N}{\mathcal{Z}} e^{\frac{N \mu - E_{N,n}}{T}}
	=
	T \sum_{N,n} \left(-\frac{N}{\mathcal{Z}^2}\right) \frac{\partial \mathcal{Z}}{\partial \mu} e^{\frac{N \mu - E_{N,n}}{T}} + T \sum_{N,n} N \frac{1}{\mathcal{Z}} \frac{N}{T} e^{\frac{N \mu - E_{N,n}}{T}}\\
	&=
	- T \frac{1}{\mathcal{Z}} \frac{\partial \mathcal{Z}}{\partial \mu} \sum_{N,n} N \frac{1}{\mathcal{Z}} e^{\frac{N \mu - E_{N,n}}{T}} + \sum_{N,n} \frac{N^2}{\mathcal{Z}} e^{\frac{N \mu - E_{N,n}}{T}}
	=
	-\underbrace{T \frac{\partial}{\partial\mu} \ln \mathcal{Z}}_{\langle N\rangle} \langle N\rangle + \langle N^2\rangle 
	= 
	- \langle N\rangle^2 + \langle N^2\rangle = \langle \Delta N^2\rangle.
\end{aligned}
\end{equation*}
Получили флуктуацию числа частиц.
А именно результат, что в большом каноническом ансамбле
\begin{equation*}
	\langle \Delta N^2\rangle = T \frac{\partial N_\text{TD}}{\partial \mu}.
\end{equation*}

Теперь остался вопрос в том, как же вычисляется $\mathcal{Z}$. Рассмотрим это на примерах.

\subsubsection*{Квантовый идеальный ферми-газ}
Пусть у нас есть газ из невзаимодействующих частиц, с уровнями энергии $\{\varepsilon_{k1}, \varepsilon_{k2}, \varepsilon_{k3}\ldots\}$. При чём на одном уровне может сидеть только по одному фермиону.
Тогда макро-состояния будут характеризоваться набором $n = (\vc{k}_1, \vc{k}_2, \vc{k}_3, \ldots, \vc{k}_N)$. Энергия каждого микро-состояния\footnote{понятно, что у свободных частиц функции состояния $\ket{\psi_{\smallvc{k},\sigma}} = \frac{1}{\sqrt{V}} e^{i \smallvc{k} \smallvc{r}}\chi_\sigma$.} это $\varepsilon_k = \frac{\hbar^2 k^2}{2 m}$.

А теперь, собственно, смотрим, как вычисляется большая статсумма
\begin{equation*}
	\mathcal{Z} = \sum_{N,n} e^{\frac{N \mu - E_{N,n}}{T}}
	=
	1 + \left(e^{\frac{\mu - \varepsilon_1}{T}} + e^{\frac{\mu - \varepsilon_2}{T}} + \ldots\right) 
	+ \left( e^{\frac{2\mu -\varepsilon_1 - \varepsilon_2}{T}} + e^{\frac{2\mu -\varepsilon_1 - \varepsilon_3}{T}} + \ldots \right) + (\ldots) + \ldots
\end{equation*}
где первое слагаемое соответствует тому, что посадили в систему $N = 0$ фермионов, второе --- $N= 1$, затем $N=2$ и так далее.

Физики-теоретики за нас посчитали чему такая сумма равна, если подставить наши наборы $n$, а мы этого выводить не будем
\begin{equation*}
	\mathcal{Z}_{FG} = \prod_{\smallvc{k}} \left(1 
	+ e^{\frac{\mu - \varepsilon_{k}}{T} }\right).
\end{equation*}
А если учитывать спин, то нужно будет ввести $l$ -- для нумерации одночастичных квантовых состояний $l(\vc{k}, \vc{\sigma})$
\begin{equation*}
	\mathcal{Z} = \prod_{l} \left(1 
	+ e^{\frac{\mu - \varepsilon_{l}}{T} }\right).
\end{equation*}
Теперь чуть более конкретизируем пример 
\subsubsection*{Газ в ящике}
Пусть $\varepsilon_l = \varepsilon_{\vc{k}, \sigma}$.
То есть 
\begin{equation*}
	\mathcal{Z}_{FG} = \prod_{\smallvc{k}, \sigma} \left(1 
	+ e^{\frac{\mu - \varepsilon_{k, \sigma}}{T} }\right).
\end{equation*}
Отсюда
\begin{equation*}
	\Omega = - T \ln \mathcal{Z} = - T \sum_{\smallvc{k}, \sigma}\ln \left(1 + e^{\frac{\mu - \varepsilon_{k,\sigma}}{T}}\right).
\end{equation*}
Теперь нам надо перейти от суммы к интегралу, при чём хочется ещё интегрировать по энергии микро-состояния $\int d k \to \int d \varepsilon_k$. А для этого нужно поговорить про плотность состояний.

\subsubsection*{Плотность состояний}
Для какой-то величины, не зависящей от координат, от суммы к интегрированию переходим следующим образом
\begin{equation*}
	\sum_{\smallvc{p}} \, f_{p} = \frac{V}{(2 \pi \hbar)^3} \int d^3 p f(\vc{p}).
\end{equation*}
При этом 
\begin{equation*}
	p_\alpha = \frac{2 \pi \hbar}{L} n_\alpha,
	\hspace{1 cm}
	\Rightarrow
	\hspace{1 cm}
	\Delta p = \frac{2 \pi \hbar}{L}.
\end{equation*}
Теперь допустим, что
\begin{equation*}
	\sum_{\smallvc{p}} f(\varepsilon(\vc{p})) = \int_0^{+\infty} \nu(\varepsilon) f(\varepsilon) d \varepsilon,
\end{equation*}
где $\nu(\varepsilon)$ --- плотность состояний.

Так например в 2-D ферми-газе
\begin{equation*}
	\varepsilon_{\smallvc{p}} = \frac{\vc{p}^2}{2 m},
	\hspace{1.5 cm}
	\sum_{\smallvc{p}} \leadsto \frac{L^2}{(2 \pi \hbar)^2} \int d^2 p.
\end{equation*}
И теперь раз $\varepsilon(|p|)$, то перейдём в полярные координаты.
\begin{equation*}
	d^2 p = 2\pi p d p,
	\hspace{1 cm}
	\varepsilon = \frac{p^2}{2m}
	\hspace{0.5 cm}
	\Rightarrow
	\hspace{0.5 cm}
	d \varepsilon =\frac{p}{m} \d p
	\hspace{0.5 cm}
	\Rightarrow
	\hspace{0.5 cm}
	m \d \varepsilon = p \d p.
\end{equation*}
Ну и вооружившись приобретенными знаниями
\begin{equation*}
	\sum_{\smallvc{p}}  f(\varepsilon(\vc{p})) 
	=  \frac{L^2}{(2 \pi \hbar)^2} \int 2\pi p  \d p \, f(\varepsilon(p)) = \frac{L^2}{(2 \pi \hbar)^2} 2 \pi m \int_0^{+\infty} \d\varepsilon \, f(\varepsilon)
	=
	\int_0^{+\infty} \nu(\varepsilon) f(\varepsilon) \d \varepsilon
\end{equation*}
Таким образом выражаем
\begin{equation*}
	\nu(\varepsilon) = \frac{2 \pi m L^2}{(2 \pi \hbar)^2},
\end{equation*}
важно что, в для двухмерного газа получилось, что $\nu_{\text{\tiny{2D}}}(\varepsilon) \sim \const$.
И ещё раз подчеркнём наш вывод с заменой суммы интегрированием
\begin{equation*}
	\sum_{\smallvc{p}} f (\varepsilon_p) 
	=
	\int \nu_{\text{\tiny{2D}}}(\varepsilon) f(\varepsilon) \d \varepsilon
	=
	\frac{2 \pi m L^2}{(2 \pi \hbar)^2} \int_0^{+\infty} f(\varepsilon) \d\varepsilon,
\end{equation*}
если теперь ещё добавим и спин\footnote{в трехмерии $\nu_\text{3D} \sim \sqrt{\varepsilon}$, но это мы пока показывать не будем.}, то
\begin{equation*}
	\nu_{\text{\tiny{2D}}} = \frac{ 2\pi g m L^2}{(2 \pi \hbar)^2},
\end{equation*}
где $g = 2 s + 1$.

И так, мы остановились на выражении для большого омега потенциала:
\begin{equation*}
	\Omega = - T \sum_{\smallvc{k}, \sigma} \ln \left(1 + e^{\frac{\mu - \varepsilon_p}{T}}\right).
\end{equation*}
Пусть мы работаем с 2D ферми-газом (не зря же мы там выше для него плотность выводили) без внешнего магнитного поля.
Поэтому
\begin{equation*}
	\Omega = - T \int_0^{+\infty} \nu_\text{\tiny{2D}} \ln \left(1 + e^{\frac{\mu - \varepsilon_p}{T}}\right) \d \varepsilon = - T \left[ \varepsilon \ln \left(1 + e^{\frac{\mu - \varepsilon_p}{T}}\right) \bigg|_0^{+\infty} + \frac{1}{T} \int_0^{+\infty} \frac{\varepsilon \d \varepsilon}{1 + (\varepsilon - \mu)/T}\right]
\end{equation*}
Итого
\begin{equation*}
	\Omega = - \int_0^{+\infty} = - \frac{\varepsilon \d \varepsilon \nu_\text{\tiny{2D}}}{1 + e^{(\varepsilon - \mu)/T}} = - E,
\end{equation*}
где
\begin{equation*}
	E = \sum_{\smallvc{p}, \sigma} \varepsilon_p \langle n | \varepsilon_p\rangle = \int_0^{+\infty} \nu_\text{\tiny{2D}} \frac{\varepsilon \d \varepsilon}{1 + e^{(\varepsilon - \mu)/T}}.
\end{equation*}