\subsection*{Флуктуации числа частиц в газе}
Возьмем $N$ частиц газа в объёме $V$. Зададимся вопросом какова вероятность найти в небольшом объёме газа $v$ ровно $n$ частиц? То есть $P_n(v) - ?$. Введём вероятность $p = v/V$ одной частицы попасть в объём в $v$. Соответственно $q = 1 - v/V$ --- это вероятность того, что частицы не находятся в этом объёме.

Тогда для $n$ частиц получим вероятность просто по биномиальному распределению
\begin{equation*}
	P_n(v) = C_N^n p^n q^{N-n}. 
\end{equation*}
Проследим как оно плавно перетекает в распределение Пуассона. Вообще биномиальный коэффициент --- достаточно сложная штука, но мы будем работать в приближении, что $v \ll V$ ($p\ll 1$), и будет нам счастье. В этом случае введём параметр $\lambda = N p$. Его физический смысл \begin{equation*}
	\lambda = N \frac{v}{V} = \frac{N}{V}v = n_\text{концентрация} \cdot v = \text{среднее число частиц в }  v = \overline{n}.
\end{equation*}
Понятно, что для параметра $\lambda$ возможны разные случаи: $\lambda < 1$, $\lambda \sim 1$ и $\lambda >1$.

Зафиксируем $\lambda$, тогда вероятности находиться и не находиться в объёме $v$
\begin{equation*}
	p^n = \left(\frac{\lambda}{N}\right)^n,
	\hspace{1 cm}
	q^{N-n} = \left(1 - \frac{\lambda}{N}\right)^{N-n}.
\end{equation*}
Вероятность тогда
\begin{equation*}
	P_n(v) = \frac{N (N-1)\ldots(N-n+1)}{n!} \frac{\lambda^n}{N^n} \frac{\left(1 - \frac{n}{N}\right)^N}{1 - \left(\frac{\lambda}{N}\right)^n}
	= \frac{\lambda^n}{n!} \frac{N(N-1) \ldots (N-n+1)}{N \cdot N \ldots N} \frac{\left(1 - \frac{\lambda}{N}\right)}{\left(1 - \frac{\lambda}{N}\right)N}.
\end{equation*}
Теперь, $n$ у нас --- фиксировано,а $N \gg 1$, тогда
\begin{equation*}
	\frac{N(N-1) \ldots (N-n+1)}{N \cdot N \ldots N} = 1 \left(1 - \frac{1}{N}\right)\ldots\left(1 - \frac{n-1}{N}\right) \longrightarrow 1, 
	\hspace{1 cm}
	N \to \infty,
\end{equation*}
\begin{equation*}
	\left(1 - \frac{\lambda}{N}\right)^n \longrightarrow 1,
	\hspace{1 cm}
	\left(1 - \frac{\lambda}{N}\right)^N \longrightarrow \exp(-\lambda).
\end{equation*}
В результате получили, что биномиальное распределение перетекло в распределение Пуассона
\begin{equation*}
	P_n(v) \longrightarrow P(n,\lambda) = \frac{\lambda^n}{n!} \exp(-\lambda).
\end{equation*}
Его характеристика 
\begin{equation*}
	\langle n\rangle = \sum_{n=0}^{\infty} n P(n,\lambda) = \lambda,
	\hspace{1 cm}
	\langle \Delta n^2\rangle = \overline{n^2} - (\overline{n})^2 = \lambda.
\end{equation*}
Откуда получаем относительную флуктуацию
\begin{equation*}
	\frac{\sqrt{\langle \Delta n^2\rangle}}{\overline{n}} = \frac{1}{\sqrt{\overline{n}}} = \frac{1}{\sqrt{\lambda}} \longrightarrow 0,
	\hspace{1 cm}
	\overline{n} \gg 1
	\hspace{2 mm}
	(\lambda \gg 1).
\end{equation*}
И можно заметить, что, при таких малых относительных флуктуациях,  распределение Пуассона переходит в распределение Гаусса
\begin{equation*}
	\varphi_{\overline{n}, \sigma} = \frac{1}{\sqrt{2 \pi \sigma^2}} \exp\left\{-\frac{(n -\overline{n})^2}{2 \sigma^2}\right\},
	\hspace{1 cm}
	\sigma^2 =\lambda.
\end{equation*}
\begin{figure}[h]
    \centering
    \includegraphics[width=0.5\textwidth]{img/sem7im1.pdf}
    %\caption{}
    %\label{fig:}
\end{figure}
Пик находится в $n = \overline{n} = \lambda$. Формально
\begin{equation*}
	P(n, \lambda) \longrightarrow \frac{1}{\sqrt{2 \pi \lambda}} \exp\left\{-\frac{(n-\lambda)^2}{2 \lambda}\right\}.
\end{equation*}
Как это делается? Сначала раскладываем в ряд $P(n,\lambda) = e^{w(n)}$ около $n = \lambda$
\begin{equation*}
	w(n) = w(\lambda) + \frac{1}{2}w''(\lambda) (n-\lambda)^2 + \ldots
\end{equation*}
У нас же $P(n,\lambda) = \frac{\lambda^n}{n!}e^{-\lambda}$, то есть
\begin{equation*}
	w(n) = \ln\left\{ - \lambda\frac{\lambda^n}{n!}\right\}
	=
	n \ln \lambda - \lambda - \ln n! = n \ln \lambda - \lambda n \ln n + n - \frac{1}{2} \ln 2\pi n,
\end{equation*}
где мы воспользовались формулой Стирлинга
\begin{equation*}
	n! \approx \left(\frac{n}{e}\right)^n \sqrt{2 \pi n},
	\hspace{1 cm}
	\Rightarrow
	\hspace{1 cm}
	\ln n! \approx n \ln n - n + \frac{1}{2} \ln 2 \pi n.
\end{equation*}
И так возможно, внимательный читатель задался вопросом, почему в разложении нет первой производной 
\begin{equation*}
	w'(n) = \ln \lambda -  \ln n - 1 + 1 - \underbrace{\frac{1}{2 n}}_\text{крайне мала},
	\hspace{1 cm}
	w'(n) = 0
	\hspace{0.5 cm}
	\Rightarrow
	\hspace{0.5 cm}
	n = \lambda.
\end{equation*}
Как видно, пик вообще узкий, поэтому там, где распределение вообще отлично от нуля, у нас везде максимум и первой производной мы пренебрегаем.
Вторая же производная будет
\begin{equation*}
	w''(n) = - \frac{1}{n}\mid_{n=\lambda} = -\frac{1}{\lambda}.
\end{equation*}
Итого получаем разложение
\begin{equation*}
	w(n) = \ln \frac{1}{\sqrt{2 \pi \lambda}} + \frac{1}{2 \lambda} (n - \lambda)^2.
\end{equation*}
Тогда 
\begin{equation*}
	e^{w(n)} = \frac{1}{\sqrt{2 \pi \lambda}} \exp \left\{-\frac{(n-\lambda)^2}{2 \lambda}\right\} = P(n,\lambda)
\end{equation*}
Получился Гаусс! Забавно, что даже при таком допущении у нас и нормировка тоже сошлась. А всё почему?
\begin{flushright}
 	 Потому что пик очень узкий. 
 \end{flushright}

\subsection*{Неравновесная энтропия}
Будем работать на примере Ферми статистики. Пусть есть уровень энергии системы $ \varepsilon$ вырожденный $g$ раз. И на этом уровне есть $n$ частиц ($0<n<g$).

Число всех микросостояний системы $=$ число способов разместить $n$ частиц по $g$ ячейкам.
\begin{equation*}
	W = C_g^n = \frac{g!}{n! (g - n)!}.
\end{equation*}

Теперь усложняем. Пусть будет $j$ уровней с энергией $\varepsilon_i$, каждый из них вырожден $g_i$ раз и на каждом $n_i$ частиц ($i = 1,\ldots,j$).
При этом число частиц на уровнях меняется, но так, чтобы сохранилось только суммарное число частиц и суммарная энергия системы
\begin{equation*}
	\sum_i n_i = N,
	\hspace{1 cm}
	\sum_i \varepsilon_i n_i = E.
\end{equation*}
Таким образом фиксируем $N, E$, что является весьма жестким ограничением для системы.

Ищем же мы $E(n_1, n_2, \ldots, n_j, \ldots) = E(\{n_j\})$.
А именно, хотим найти кратность вырождения с заданным фиксированным набором ${n_j}$ энергии $E \equiv$ статвесу макросостояния $(N,E)$. $W(N,E) - ?$
\begin{equation*}
	W = \prod_{j} W_j,
\end{equation*}
где $j$ -- нижний индекс по (??).
$\{n_j\}$ -- не просто любой набор,но ещё и удовлетворяющий закону сохранению числа частиц и энергии.

Пусть мы нашли кратность вырождения уровня с $E$, так что все ${n_j}$ -- фиксированные, и \textbf{неравновесная энтропия}
\begin{equation*}
	S = \ln W(N, E, {n_j}).
\end{equation*}
Здесь нужно, подчеркнуть, что $S(\{n_j\})$ -- энтропия именно от набора, который доставляет ей максимум. 
И тогда, если $S(N,E, \{n_j\})$ $\Rightarrow$ $S$ будет равновесной термодинамической энтропией $S(E,N)$ (набор фиксированный).

Все наши рассуждения упирались в то, что набор $\{n_j\}$ должен быть достаточно удачный, найдём же такой набор, что доставит максимум энтропии. Если посмотреть выше, то видно, что всё уже у нас для поиска этого набора есть
\begin{equation*}
	S = \sum_j = \big[\ln g_j ! - \ln n_j ! - \ln(g_j - n_j)!\big]
		=
		\sum_j \big[g_j \ln g_j - n_j \ln n_j - (g_j - n_j) \ln(g_j - n_j)\big]
\end{equation*}
И не забываем, что выполнятся должны
\begin{equation*}
	\sum_j n_j = N,
	\hspace{0.5 cm}
	\hspace{0.5 cm}
	\sum_j \varepsilon_j n_j = E.
\end{equation*}
Чтобы найти такие $n_j$ которые доставляют энтропии максимум и учесть условия, воспользуемся формулой Лагранжа
\begin{equation*}
	\tilde{S} = S + \alpha N + \beta E
	\hspace{1 cm}
	\leadsto
	\hspace{1 cm}
	\frac{\partial \tilde{S}}{\partial{n_j}} = 0.
\end{equation*}
По отдельности
\begin{equation*}
	\frac{\partial S}{\partial n_j} = - \ln n_j - 1 + \ln(g_j - n_j) + 1 = - \ln \frac{n_j}{g_j	- n_j},
	\hspace{1 cm}
	\frac{\partial N}{\partial n_j} = 1,
	\hspace{1 cm}
	\frac{\partial E}{\partial n_j} = \varepsilon_j.
\end{equation*}
Таким образом получаем
\begin{equation*}
	- \ln \frac{n_j}{g_j - n_j} + \alpha + \beta \varepsilon_j = 0
	\hspace{1 cm}
	\Rightarrow
	\hspace{1 cm}
	\frac{n_j}{g_j - n_j} = e^{\alpha + \beta \varepsilon_j} = \chi,
\end{equation*}
далее
\begin{equation*}
	n_j = g_j \chi - n_j \chi 
	\hspace{0.5 cm}
	\Rightarrow
	\hspace{0.5 cm}
	(1 + \chi) n_j = g_j \chi
	\hspace{0.5 cm}
	\Rightarrow
	\hspace{0.5 cm}
	n_j = g_j \frac{\chi}{1 + \chi} = g_j \frac{e^{\alpha + \beta \varepsilon_j}}{1 + e^{\alpha + \beta \varepsilon_j}}.
\end{equation*}
Вот именно такие числа $n_j$ заполнения доставляют максимум энтропии.
Для окончательной радости нам остаётся только найти $\alpha$ и $\beta$.

Сравним полученную формулу Лагранжа с известным нам дифференциалом
\begin{equation*}
	d S + \alpha d N + \beta d E = 0
	\hspace{1 cm}
	\longleftrightarrow
	\hspace{1 cm}
	d S + \frac{\mu}{T} d N - \frac{1}{T} d E - \frac{P}{T} d V = 0.
\end{equation*}
Раз мы в задаче фиксируем уровни $\varepsilon_j$, то они, являясь функцией $V$ (частицы в ящике) фиксируют и объём: $V = \const$ $\Rightarrow$ $d V = 0$.

Тогда не трудно сравнить оставшиеся члены и понять, что
\begin{equation*}
	\alpha = \frac{\mu}{T},
	\hspace{1.5 cm}
	\beta = - \frac{1}{T}.
\end{equation*}
Для $S$ термодинамической энтропии хоть вынь хоть положь, эти частицы. И ей же и является доставленная в максимум статистическая энтропия $d S + \alpha d N + \beta d E$.

Подставляя коэффициента $\alpha, \beta$ получаем что 
\begin{equation*}
	n_i = g_i \frac{1}{e^{\frac{\varepsilon_i - \mu}{T}}+1}
	\hspace{1 cm}
	\leadsto
	\hspace{1 cm}
	\frac{n_i}{g_i} = f_i = \frac{1}{e^{\frac{\varepsilon_i - \mu}{T}}+1},
\end{equation*}
где $f_i$ -- числа заполнения $i$-го уровня. А вся эта красота называется распределением Ферми-Дирака.

А теперь остаётся просуммировать по всем наборам.
Среди этих сумм одно слагаемое сильно перевешивает все остальные. Оно и будет давать главный вклад в энтропию, прямо как в начале семинара, где мы рассуждали про острый пик.