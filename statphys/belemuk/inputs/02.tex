\subsubsection*{Адиабатическое размагничивание (задача)}
Имеем кристаллическую решетку с примесями. (Магнито-каллорический эффект)
\begin{equation*}
	E = E_{\text{поле решетки}} + E_{\text{примесей в малом поле}},
\end{equation*}
если $H$ -- понижаем, то $T$ образца уменьшается. Нужно найти $\frac{\partial T}{\partial H} - ?$.

\subsubsection*{Будем решать.}

Функцией чего является энергия? $E(S,V H)$. Пренебрегаем изменением $V$, то есть объём фиксирован.
\begin{equation*}
	d E = T d S - M d H
\end{equation*}
тогда
\begin{equation*}
	\left(\frac{\partial T}{\partial H}\right)_{S} = - \left(\frac{\partial M}{\partial S}\right)_{H} 
	= - \frac{\partial (M, H)}{\partial (T, H)}
	= - \frac{\partial(M, H)}{\partial(T,H)} \frac{\partial(T, H)}{\partial(S,H)}
	= - \left(\frac{\partial M}{\partial T}\right)_H \cdot \frac{T}{T (\partial S/\partial T)_H} 
	= \left(- \frac{\partial M}{\partial T}\right)_H \frac{T}{C_{V, H}}
\end{equation*}
Кроме того теплоёмкость системы, примесная же теплоёмкость это
\begin{equation*}
	C = C_\text{решетки} + C_\text{примесей} \approx C_{V, H} \approx \alpha T^3.
\end{equation*}
Парамагнетик в слабых полях (закон Кюри): 
$$M(T) = \chi(T) H, \hspace{1 cm}\chi(T) = A/T.$$ Таким образом
\begin{equation*}
	\left(\frac{\partial M}{\partial T}\right)_H = \chi'(T) H;
	\hspace{1 cm}
	\left(- \frac{\partial M}{ \partial T}\right)_H = \frac{A}{T^2} H
\end{equation*}
Таким образом 
\begin{equation*}
	\left(\frac{\partial T}{ \partial H}\right)_S = \frac{A}{T^2} H \frac{T}{\alpha T^3} = \frac{A H}{\alpha T^4}.
\end{equation*}
Следовательно, 

Если $H$ увеличиваем, то $ T$ --- увеличивается. Если $H$ уменьшается, то $T$ --- уменьшается.

\subsubsection*{Двухуровневая Система}
Имеем $N$ атомов, часть из них $n$ на возбужденном уровне. Тогда задана и энергия системы
\begin{equation*}
	E_{\text{сист}} = 0 \cdot (N - n) + n \varepsilon = n \varepsilon.
\end{equation*}
Наша задача найти $S(E) - ?$ (и всё так далее про термодинамику)

И так, задание $(N, n)$ однозначно определяет $(N, E)$ и наоборот. Энтропия будет функцией $S(E,N)$.
И по Больцману 
\begin{equation*}
	S = \ln W(E),
\end{equation*}
где $W(E)$ -- статистический вес -- число микроскопических состояний, отвечающих заданному макросостоянию.
Число таких микросотояний легко посчитать $W(E) = C_N^n = \frac{N!}{n! (N-n)!}$.

По формуле Стирлинга
\begin{equation*}
	N! \approx \left(\frac{N}{e}\right)^{N}, 
	\hspace{1 cm}
	\ln N! \approx N \ln N - N.
\end{equation*}
Тогда
\begin{align*}
	\ln C_N^n &\approx \ln N! - \ln n! = \ln(N-n)! = N\ln N - N - n \ln n + n - (N-n) \ln(N-n) + N - n \\
	&= N \ln N - n \ln n - (N-n) \ln N - (N-n) \ln\left(1- \frac{n}{N}\right) \\
	& = - n \ln\frac{n}{N} - N \left(1 - \frac{n}{N}\right) \ln\left(1 - \frac{n}{N}\right) \\
	& = N \left[- \frac{n}{N} \ln \frac{n}{N} - \left(1 - \frac{n}{N}\right) \ln \left(1 - \frac{n}{N}\right)\right] = S
\end{align*}
Энтропия у нас пропорциональная $N$, а в скобках там стоит плотность $n/N = x$ -- доля возбужденных атомов.
\begin{equation*}
	E = \varepsilon n 
	\hspace{1 cm}
	\frac{n}{N} = \frac{E}{\varepsilon N},
\end{equation*}
тут удобно ввести $E/N$ -- удельная энергия, $s = S/N$ -- удельная энтропия. Тогда
\begin{equation*}
	S = N s(\frac{E}{N}), 
	\hspace{1 cm}
	S = N s(x).
\end{equation*}

Посмотрим на выводы. Имеем функцию
\begin{equation*}
	s(x) = - x \ln(x) - (1-x) \ln(1-x),
	\hspace{1 cm}
	0 \leq x \leq 1.
\end{equation*}
\red{построим график}

Мы помним, что S(E) -- всегда растущая функция $E$ (в обычных системах), выпуклая вверх (convex).
А на графике мы видим, что есть ветвь, которая хоть и выпуклая вверх, но убывает.
Это связано с тем, что мы рассматриваем спиновую систему с конечным числом уровней. Тогда энергия системы имеет ограничение
\begin{equation*}
	0 \leq E \leq E_{\text{max}} = N \varepsilon,
\end{equation*}
а число возможных квантовых состояний конечно $\equiv 2^{N}$ и в итоге в тиках системах и появляется ветвь с убывающей энтропией.

На лекции мы узнали, что 
\begin{equation*}
	\left(\frac{\partial S}{\partial E}\right)_{N} = \frac{1}{T},
\end{equation*}
значит на той ветви температура отрицательна. Найдём её
\begin{equation*}
	\frac{\partial S}{\partial E} = \frac{\partial S/N}{\partial E /N} = \frac{ \partial s}{\partial \varepsilon x} = \frac{1}{\varepsilon} \frac{\partial s}{\partial x}.
\end{equation*}
Функция $s(x)$ задана выше, её производная легко находится
\begin{equation*}
	s'(x) = ln \frac{1 - x}{x}
	\hspace{1 cm}
	\Rightarrow
	\hspace{1 cm}
	\frac{\partial S}{\partial E} = \frac{1}{\varepsilon} \ln \frac{1 - x}{x} = \frac{1}{T}
	\hspace{1 cm}
	\Rightarrow
	\hspace{1 cm}
	T(x) = \varepsilon \frac{1}{\ln\left(\frac{1 - x}{x}\right)}.
\end{equation*}
\red{построим график}

Почему же температура при $x = 1/2$ уходит в бесконечность? Как вообще в система (например) с двумя положениями спинов почувствовать? Давайте посчитаем что происходит в нашей системе с возбуждающимися атомами.

Найдём зависимость от температуры
\begin{equation*}
	x(T) =  \frac{1}{1 + e^{\varepsilon/T}}.
\end{equation*}
\red{построим график}

Выглядит как ферми ступенька, которая нигде не кончится. Физически же это означает, что как бы мы ни грели систему больше чем половину уровня мы не заполним.
Или если, посмотреть на область $T<0$, то они как бы "горячее" (заселеннее) чем $T = + \infty$ (они обе соответствуют заселенности $x>1/2$).
На эксперименте это реализуется с помощью переворота магнитного поля в системе спинов, и тогда верхний уровень будет заселён как раз больше.

Любопытствующий студент может спросить про трёхуровневую систему
\begin{equation*}
	\frac{n_3}{n_2} \sim \frac{e^{\varepsilon_3/T}}{e^{\varepsilon_2/T}},
\end{equation*}
но тут нужно быть осторожным и всё аккуратно посчитать.

Осталось найти среднюю энергию
\begin{equation*}
	\langle E\rangle = \varepsilon \langle n\rangle(T) = \frac{\varepsilon N}{1 + e^{\varepsilon/T}}
	\hspace{1 cm}
	\Rightarrow
	\hspace{1 cm}
	\langle E\rangle_{T \to \infty} = \frac{\varepsilon N}{2}.
\end{equation*}
А теперь теплоёмкость
\begin{equation*}
	C(T) = \frac{d E}{d T} = \frac{\varepsilon^2}{T^2} N \frac{e^{\varepsilon/T}}{\left(1 + e^{\varepsilon/T}\right)^2},
	\hspace{1 cm}
	c = \frac{C}{N} = \frac{\varepsilon^2}{T^2}\frac{e^{\varepsilon/T}}{\left(1 + e^{\varepsilon/T}\right)^2}.
\end{equation*}
\red{построим график}

\noindent Важно отметить три особенности
\begin{enumerate}
	\item $T \to 0$, то и $c \to 0$, теплоёмкость экспоненциально мала;
	\item $c$ -- имеет максимум, если число уровней ограничено, значит он есть (теплоёмкость Шотки);
	\item $T \to \infty$, то $c \sim 1/T^2$.
\end{enumerate}
