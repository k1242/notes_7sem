% Макроскопические квантовые явления
% Введение в сверхтекучесть. Критерий сверхтекучести Ландау. Бозе-эйнштейновская конденсация в идеальном бозе-газе. Теория сверхтекучести Боголюбова для слабо-неидеального газа.

\subsection*{Критерий сверхтекучести Ландау}

Lorem ipsum dolor sit amet, consectetur adipisicing elit, sed do eiusmod
tempor incididunt ut labore et dolore magna aliqua. Ut enim ad minim veniam,
quis nostrud exercitation ullamco laboris nisi ut aliquip ex ea commodo
consequat. Duis aute irure dolor in reprehenderit in voluptate velit esse
cillum dolore eu fugiat nulla pariatur. Excepteur sint occaecat cupidatat non
proident, sunt in culpa qui officia deserunt mollit anim id est laborum.

\subsection*{Идеальный бозе-газ}

Lorem ipsum dolor sit amet, consectetur adipisicing elit, sed do eiusmod
tempor incididunt ut labore et dolore magna aliqua. Ut enim ad minim veniam,
quis nostrud exercitation ullamco laboris nisi ut aliquip ex ea commodo
consequat. Duis aute irure dolor in reprehenderit in voluptate velit esse
cillum dolore eu fugiat nulla pariatur. Excepteur sint occaecat cupidatat non
proident, sunt in culpa qui officia deserunt mollit anim id est laborum.




\subsection*{Неидеальный Бозе-газ}

Рассмотрии гамильтониан
\begin{equation*}
	H = \sum_p \frac{p^2}{2m} a_p\con a_p + \frac{g}{2V} \sum a_{p_1'}\con a_{p_2'}\con a_{p_2} a_{p_1},
\end{equation*}
где $\sum \sub{p}{in} = \sum \sub{p}{out}$. 

Основной вклад идёт от нулевых импульсов:
\begin{equation*}
	 \frac{g}{2V} \sum a_{p_1'}\con a_{p_2'}\con a_{p_2} a_{p_1} \approx \frac{g}{2V} a_0\con a_0\con a_0 a_0 + \frac{g}{2V} \sum_p \left(
		a_0\con a_0\con a_p a_{-p} + a\con_{p} a_{-p}\con a_0 a_0 + 4 a_0\con a_0 a_p\con a_p
	\right),
\end{equation*}
где первое слагаемое пропорционально
\begin{equation*}
	N_0^2 = (N - \sum_p a_p\con a_p)^2 = N - 2 N \sum_p a_p\con a_p.
\end{equation*}
Итого находим\footnote{
	Вообще около $a_0\con$ могла быть фаза, но от неё каноническим преобразованием можем избавиться.
} , что
\begin{equation*}
	H = \frac{g N^2}{2V} + \sum_p \left[
		\left(\frac{p^2}{2m} + \frac{2 g N}{2V}\right) a_p\con a_p + 
		\frac{g N}{2V} \left(
			a_p\con a_{-p}\con + a_p a_{-p}
		\right)
	\right],
\end{equation*}
таким образом приходим к квадратичному гамильтониану. 


Сделаем каноническое преобразование Боголюбова
\begin{equation*}
	\left\{\begin{aligned}
	    a_p &= u_p b_p + v_p b\con_{-p} \\
	    a_p\con &= u_p b_p\con + v_p b_{-p}\con
	\end{aligned}\right.
	\hspace{10 mm} 
	[b_p,\, b_{p'}\con] = \delta_{pp'},
	\hspace{5 mm} 
	[b_p,\, b_{p'}] = 0.
\end{equation*}
Так приходим к ограничениям вида 
\begin{equation*}
	u_p^2 - v_p^2 = 1,
	\hspace{10 mm} 
	v_{p} = v_{-p}, 
	\hspace{10 mm} 
	u_{p} = u_{-p},
\end{equation*}
более того можем выбрать коэффициенты вещественными.  

Из диагонализации, с учетом требований, получаем выражение для $u_p$ и $v_p$:
\begin{equation*}
	u_p = \frac{1}{2}\left(
		\sqrt{\frac{\varepsilon(p)}{E_p}} + \sqrt{\frac{E_p}{\varepsilon(p)}}
	\right),
	\hspace{10 mm} 
	u_p = \frac{1}{2}\left(
		\sqrt{\frac{\varepsilon(p)}{E_p}} - \sqrt{\frac{E_p}{\varepsilon(p)}}
	\right),
\end{equation*}
где энергия $\varepsilon(p)$ имеет вид
\begin{equation*}
	\varepsilon(p) = \sqrt{s^2 p^2 + \left(\frac{p^2}{2m}\right)^2},
	\hspace{10 mm} 
	s = \sqrt{\frac{g n }{m}}.
\end{equation*}
Таким образом гамильтониан приводится к виду
\begin{equation*}
	H = H_0 + \sum_{p \neq 0} \varepsilon(p) b_p\con b_p.
\end{equation*}
Таким образом боголюбоны соответсвуют идеальному бозе-газу возмущений. 

% \begin{equation*}
% 	\langle b_p\con b_p \rangle = \frac{1}{e^{\varepsilon(p)/T}-1}.
% \end{equation*}

По критерию Ландау сразу находим
\begin{equation*}
	\sub{v}{кр} = \min_p \frac{\varepsilon(p)}{p} = s.
\end{equation*}


Найдём количество надконденсатных частиц
\begin{equation*}
	N_{p\neq 0} = \sum_{p\neq 0} \langle a_p\con a_p\rangle = \sum_{p\neq 0} \left(
		u_p^2 \langle b_p\con b_p\rangle + v_p^2 (1-\langle b_p\con b_p\rangle)
	\right) \overset{T=0}{=}  \sum_p v_p^2 \sim V \cdot (ms)^3.
\end{equation*}
Длину рассеяния $a$ можем вычислить для $s$-рассеяния
\begin{equation*}
	a = \frac{gm}{4 \pi}.
\end{equation*}
Подставляя $a$ в выражение для $N_>$, находим
\begin{equation*}
	N_{p\neq 0} \sim N \cdot \sqrt{n a^3},
	\hspace{10 mm} 
	\frac{N_{p\neq 0}}{N} \sim \sqrt{n a^3},
	\hspace{0.5cm} \Rightarrow \hspace{0.5cm}
	N \approx N_0 (1 + \ldots \sqrt{n a^3}).
\end{equation*}
Для введение парных столкновений требуем, чтобы $n a^3 \ll 1$. 
