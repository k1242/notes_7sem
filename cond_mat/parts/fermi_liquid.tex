% Концепция Ландау Ферми-жидкости.

% Кулоновское взаимодействие в металлах, 
% экранирование внешнего потенциала (приближение Томаса-Ферми)

% Диэлектрическая функция в металлах.
%  Плазмоны.

% уменьшить количество слоёв
% увеличить вклад по времени
% 

\subsection{Концепция Ферми-жидкости Ландау}


Считаем, что рассеиваться могут только электроны вблизи поверхности Ферми, говорим про квазиэлектрон и квазидырку вблизи поверхности Ферми. Также считаем, что время рассеяния велико $T \sim \hbar/\tau \ll \eF$. 

Для концентрации электронов можем найти
\begin{equation*}
	n = 2 \frac{1}{(2\pi)^3} \int_{k < \kF} \d^3 k = \frac{\kF^3}{3 \pi^2}.
\end{equation*}
Вводя характерный размер -- боровский радиус $\aB$, можем заметить что при $\kF \aB \ll 1$ получится вигнеровский кристалл, а при $\kF \aB \gg 1$ -- Ферми-газ. 



\subsection{Экранирование Томаса-Ферми}

Рассмотрим действие некоторого внешнего потенциала в системе Ферми-газа. Будем считать, что $\kF (r) = \kF + \delta \kF(r)$. Рассматриваем линейный отклик системы на воздействие внешнего потенциала $\delta U_e (r)$. Это приведёт к некоторому эффективному потенциалу $\delta U(r) = \delta U_e (r) + e \varphi$, который и будут чувствовать электроны
\begin{equation*}
	\mu = \frac{\kF^2}{2m} + \delta U(r) = \const,
	\hspace{0.5cm} \Rightarrow \hspace{0.5cm}
	\delta \kF(r) = - \delta U(r) \frac{m}{\kF}.
\end{equation*}
Вспоминая, что $n = \frac{\kF^3}{3 \pi^2}$, находим
\begin{equation*}
	\delta n = \frac{\kF^2}{\pi^2} \delta \kF = - m \frac{\kF}{\pi^2} \delta U(r).
\end{equation*}
Для плотности заряда $\delta \rho = e \delta n$, тогда
\begin{equation*}
	\left\{\begin{aligned}
	    \div \vc{E} &= 4 \pi \rho \\
	    \vc{E} &= - \vc{\nabla} \varphi
	\end{aligned}\right.
	\hspace{0.5cm} \Rightarrow \hspace{0.5cm}
	\nabla^2 \varphi(r) = - 4 \pi e \, \delta n = \frac{4 e m \kF}{\pi} \delta U(r),
\end{equation*}
где добавку $\delta U(r)$ распишем в виде $\delta U(r) = \delta U_e (r) + e \varphi(r)$. Таким образом находим уравнение на вклад потенциал $\varphi(r)$:
\begin{equation*}
	\left(
		\nabla^2 - \frac{4 e^2 \kF^2}{\pi \vF}
	\right)\varphi(r) = \frac{4 e \kF^2}{\pi \vF} \, \delta U_e (r).
\end{equation*}

Переходя к Фурье образу $\varphi(r) = \frac{1}{V} \sum_q e^{i q r} \varphi(q)$, находим
\begin{equation*}
	(-q^2-\TF^2) \varphi(q) = \frac{\TF^2}{e}  \delta U_e(q),
\end{equation*}
таким образом получаем искомое экранирование потенциала:
\begin{equation*}
	\varphi(q) = - \frac{\TF^2}{e} \frac{\delta U_e (q)}{q^2 + \TF^2},
	\hspace{10 mm} 
	\TF^2 = \frac{4 e^2 m}{\pi} \kF = \frac{4}{\pi} \frac{\kF}{\aB},
\end{equation*}
где $\TF$ -- импульс Томаса-Ферми. 

Считая потенциал кулоновским $\delta U_e (r) = e^2/r$ и $\delta U_e (q) = 4 \pi e^2/q^2$, находим
\begin{equation*}
	\delta U (q) = \delta U_e (q) \left(1 - \frac{\TF^2}{q^2 + \TF^2}\right) = \frac{4 \pi e^2}{q^2 + \TF^2},
	\hspace{0.5cm} \Rightarrow \hspace{0.5cm}
	U(r) = \frac{e^2}{r} e^{-r/\rTF},
\end{equation*}
где $\rTF = 1/\TF$. Критерием применимости приведенного выше вывода является достаточная гладкость потенциала $\kF \rTF \gg 1$ или $\kF \aB \gg 1$, что выполняется. 





\subsection{\redx Диэлектрическая функция в металлах}

Lorem ipsum dolor sit amet, consectetur adipisicing elit, sed do eiusmod
tempor incididunt ut labore et dolore magna aliqua. Ut enim ad minim veniam,
quis nostrud exercitation ullamco laboris nisi ut aliquip ex ea commodo
consequat. Duis aute irure dolor in reprehenderit in voluptate velit esse
cillum dolore eu fugiat nulla pariatur. Excepteur sint occaecat cupidatat non
proident, sunt in culpa qui officia deserunt mollit anim id est laborum.


\subsection{\redx Плазмоны}

Lorem ipsum dolor sit amet, consectetur adipisicing elit, sed do eiusmod
tempor incididunt ut labore et dolore magna aliqua. Ut enim ad minim veniam,
quis nostrud exercitation ullamco laboris nisi ut aliquip ex ea commodo
consequat. Duis aute irure dolor in reprehenderit in voluptate velit esse
cillum dolore eu fugiat nulla pariatur. Excepteur sint occaecat cupidatat non
proident, sunt in culpa qui officia deserunt mollit anim id est laborum.






% Объём поверхностного слоя $4 \pi^2 \kF^2 \cdot T$. 

