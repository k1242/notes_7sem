
%+ Параметр порядка и теория Ландау фазовых переходов. 
%+ Флуктуации. ?Голдстоуновские моды.  
%+ Критерий Гинзбурга-Леванюка, верхняя критическая размерность.
% Фазовый переход в цепочке Изинга в поперечном поле.

\subsection{Теория Ландау}


Рассмотрим теорию  Ландау фазовых переходов II рода. 
Переход II рода характеризуется разрывом $\mu''$ и изменением симметрии: более симметричная фаза обычно соответсвует более высоким температурам $T > \Tc$, менее симметричная фаза соответственно при $T < \Tc$, гле $\Tc$ -- температура перехода.

Введем \textit{параметр порядка} $\eta$ так, чтобы в несимметричной фазе $\eta(T>\Tc) \equiv 0$ и $\eta(T < \Tc) \neq 0$. Вблизи от точки перехода $\eta$ принимает малые значения, тогда можем разложить, например, свободную энергию. Считая $F(\eta) = F(-\eta)$\footnote{
	Это почти всегда так, нам интересна намагниченность в качестве $\eta$, так что точно можем нечетные степени опустить.
} :
\begin{equation*}
	F = F_0 + \frac{a}{2} \eta^2 + \frac{b}{4} \eta^4 + \ldots
\end{equation*}
где $b > 0$.


В излагаемой теории предполагается, что $A(T)$ не имеет особенности в точке перехода, так что вблизи нее она разложима по целым степеням <<расстояния>> до этой точки
\begin{equation*}
	a(T) = \alpha (T-\Tc).
\end{equation*}
Из условия минимальности $F(\eta)$, находим
\begin{equation*}
	\eta^2 = -\frac{a}{b} = \frac{\alpha}{b}(\Tc - T),
	\hspace{0.5cm} \Rightarrow \hspace{0.5cm}
	\boxed{
		\eta = \pm \sqrt{\tfrac{\alpha}{b} (\Tc-T)}
	}
\end{equation*}
при $T < \Tc$ и $\eta \equiv 0$ при $T > \Tc$. 

Сразу можем найти, например, скачок теплоёмкости
\begin{equation*}
	S = - \frac{\partial F}{\partial T},
	\hspace{5 mm} 
	C = T \frac{\partial S}{\partial T},
	\hspace{0.5cm} \Rightarrow \hspace{0.5cm}
	C = C_0 + \frac{\alpha^2}{b}\Tc.
\end{equation*}

Теория Ландау применима в области немного отстоящей от $\Tc$, так как в $\Tc$ доминирует вклад от флуктуаций, но при этом $|\Tc - T| \ll \Tc$  для малости $\eta$. 



\subsection{Теория Гинзбурга-Ландау}
Вообще $\eta$ можем флуктуировать по объёму, тогда корректнее рассматривать $\eta(r)$, и $F[\eta(r)]$ -- функционал\footnote{
	Подробнее про отсутствие других производных от $\eta$ можно прочитать в ЛЛ5 \S 146.
}  от параметра порядка:
\begin{equation*}
	F[\eta(r)] = \int d^D r\ 
	\left(
		\frac{\alpha (T-\Tc)}{2} \eta^2 + \frac{b}{4} \eta^4 + C (\nabla \eta_r)^2
	\right),
\end{equation*}
где $C$ отвечает за ограничение роста числа флуктуаций. Соответсвенно при $C \to \infty$ получалась бы теория Ландау: $\eta(r) = \const$. 

Соберем из доступных констант две длины:
\begin{equation*}
	\rc = \sqrt{\frac{C}{\alpha(T-\Tc)}},
	\hspace{10 mm} 
	r_0 = \left(\frac{b \Tc}{\alpha^2 (T-\Tc)^2}\right)^{1/D},
\end{equation*}
где $\rc$ -- корреляционный радиус. Таким образом можем игнорировать вклад от флуктуаций при $\rc \gg r_0$. 

В размерности $D = 3$:
\begin{equation*}
	 \frac{T-\Tc}{\Tc} \gg \text{Gi} = \frac{b^2 \Tc}{\alpha C^3},
\end{equation*}
где Gi -- \textit{число Гинзбурга}, собственно это и образует критерий Гинзбурга-Леванюка.
При $D = 5$:
\begin{equation*}
	\frac{T-\Tc}{\Tc} \ll \frac{\Tc C^5}{\alpha b^2},
\end{equation*}
таким образом $D = 5$ -- верхняя критическая размерность, теория Ландау применима в $\Tc$ для $D > 4$. 


\subsection{Внешнее поле}

% (согласно теореме о дифференцировании по параметру— ср.
% A1.4), A5.11)).
Внешнее поле $h$ даст добавку в $F$:
\begin{equation*}
	F = F_0 + \frac{\alpha}{2} (T-\Tc) + \frac{b}{4} \eta^4 - \eta h V,
\end{equation*}
получается поле понижает симметрию более симметричной фазы, так что разница между обеими фазами исчезает -- переход <<размывается>>. 


Конкретизируем рассмотрение к модели Изинга, тогда параметром порядка $\eta$ будет удельная намагниченность $M$, внешнее поле $B$:
\begin{equation*}
	\frac{\partial F}{\partial M} = \alpha (T-\Tc) M + b M^3 - B = 0.	
\end{equation*}
Появляется дополнительная намагниченность к спонтанной $M = M_0 + \delta M$, в небольших полях $\delta M = \chi B$ и получаем закон Кюри-Вейса для восприимчивости
\begin{equation*}
	\frac{1}{\chi(T)} = \left\{\begin{aligned}
	    &2 \alpha (\Tc - T), &T < \Tc, \\
	    & \alpha (T-\Tc), &T > \Tc,
	\end{aligned}\right.
\end{equation*}
где учли, что при $T > \Tc$ спонтанная намагниченность $M_0 = 0$. 