% Атомные масштабы и малые параметры в теории конденсированных сред.
% Частицы и квазичастицы. 
% Модель желе. 
% Метод функционала плотности.

\subsection*{Модель желе}

Рассмотрим модель желе: узлы решетки + электроны в целом электронейтральны. Характерные величины системы:
\begin{equation*}
	[e^2] = \frac{\text{г}\,\text{см}^3}{\text{c}^2},
	\hspace{5 mm} 
	[m] = \text{г},
	\hspace{5 mm} 
	[n] = \text{см}^{-3}, 
	\hspace{5 mm} 
	[\hbar] = \frac{\text{г}\,\text{см}^2}{\text{c}}.
\end{equation*}
Из них можем составить две характерные энергии
\begin{equation*}
	\sub{E}{C} = e^2 n^{1/3},
	\hspace{10 mm} 
	\sub{E}{F} = \frac{\hbar^2}{m} n^{2/3}.
\end{equation*}
Вспомним, что боровский радиус $a_0 = \hbar^2 / m e^2$.
% , постоянная тонкой структуры $\alpha = e^2 / \hbar c$. 
При $n \gg a_0^{-3}$: $\sub{E}{F} \gg \sub{E}{C}$ -- получается Ферми-газ. При $n \ll a_0^{-3}$: $\sub{E}{F} \ll \sub{E}{C}$ -- вигнеровский кристалл.

\subsection*{Метод функционала плотности}

Рассмотрим гамильтониан
\begin{equation*}
	\hat{H} = \sum_j \frac{p_j^2}{2m} + \sum_j U(r_j) + e^2 \sum_{i \neq j} \frac{1}{|r_i - r_j|}.
\end{equation*}
Собственно потенциал $U(r)$ задаёт волновую функцию и концентарцию $n(r)$. 

\begin{to_thr}[Hohenberg-Kohn I]
    По концентрации $n(r)$ однозначно восстанавливается $U(r)$.
\end{to_thr}

\begin{to_thr}[Hohenberg-Kohn II]
    Для заданной $n(r)$ существует такой $\tilde{U}(r)$, что система невзаимодейтсвующих частиц с гамильтонианом $\tilde{H} = \sum_j \frac{p_j^2}{2m} + \sum_j \tilde{U}(r_j)$ приходит к плотности $n(r)$. 
\end{to_thr}

Остаётся вопрос как найти $\tilde{U}(r)$. Понятно, что
\begin{equation*}
	\tilde{U}(r) = U(r) + e^2 \int \frac{n(r')}{|r-r'|} \d^3 r' + V(r),
\end{equation*}
где в рамках local density approximation (LDA) утверждаем, что $V(r) = V(r)[n(r)]$ -- зависит только от концентрации а той же точке. 