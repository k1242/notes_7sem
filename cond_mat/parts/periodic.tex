
\subsection{Квазиимпульс и теорема Блоха}


\textbf{Квазиимпульс}.
Вспомним, что импульс -- генератор трансляций по координате $\hat{T}_{a}^x$:
\begin{equation*}
	\hat{T}_{a}^x \psi(x) = \exp\left({a}\, \partial_x \right) \psi(x) = \sum_{n=0}^{\infty} \frac{\psi^{(n)}(x)}{n!} a^n= \psi(x+{a}),
	\hspace{10 mm} 
	\hat{T}_{a}^x = \exp\left(\tfrac{i}{\hbar} {a} \hat{p} \right),
	\hspace{5 mm} 
	\hat{p} = - i \hbar \partial_x.
\end{equation*}
Верно и в обратную сторону, для импульса оператор $\hat{x}$ будет выступать генератором трансляций
\begin{equation*}
	\hat{T}^p_b = \exp\left(\tfrac{i}{\hbar} b \hat{x}\right),
	\hspace{5 mm} 
	\hat{x} = i \hbar \partial_p,
\end{equation*}
в импульсном представлении.


Рассмотрим периодический потенциал $U(x) = \sum_n e^{2 \pi n x/a}$ с периодом $a$ и заметим, что каждое слагаемое -- трансляция импульса на величину $2 \pi \hbar / a$, тогда определим квазиимпульс $P$
\begin{equation*}
	P = p \ \text{mod} \ \frac{2 \pi \hbar}{a},
\end{equation*}
который будет коммутировать с $\hat{U}$, а значит и с $\hat{H}$, то есть будет первым интегралом системы.

\textbf{Теорема Блоха}. Для $\hat{H} = \frac{\hat{p}^2}{2m} + U(r)$ c периодичным потенциалом $U$ собственные состояния могут быть выбраны в виде 
\begin{equation*}
	\psi_{n \vc{k}} = e^{i \vc{k} \vc{r}} u_{n \vc{k}}(\vc{r}),
	\hspace{10 mm} 
	u_{n \vc{k}} (\vc{r} + \vc{R}) = u_{n \vc{k}} (\vc{r}),
\end{equation*}
где $\vc{R}$ -- вектор решётки Бравэ.


\subsection{ \greenx Зонная структура}

Lorem ipsum dolor sit amet, consectetur adipisicing elit, sed do eiusmod
tempor incididunt ut labore et dolore magna aliqua. Ut enim ad minim veniam,
quis nostrud exercitation ullamco laboris nisi ut aliquip ex ea commodo
consequat. Duis aute irure dolor in reprehenderit in voluptate velit esse
cillum dolore eu fugiat nulla pariatur. Excepteur sint occaecat cupidatat non
proident, sunt in culpa qui officia deserunt mollit anim id est laborum.

