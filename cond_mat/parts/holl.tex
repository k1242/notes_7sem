% Квантовые эффекты в проводниках
% Ограничения полуклассического описания электронного транспорта уравнением Больцмана. Критерий Иоффе-Регеля.
% Локализация Андерсона в сильном беспорядке.
% Формула Ландауера для кондактанса.
% Введение в квантовый эффект Холла.  








\subsection{\greenx Локализация Андерсона}


Lorem ipsum dolor sit amet, consectetur adipisicing elit, sed do eiusmod
tempor incididunt ut labore et dolore magna aliqua. Ut enim ad minim veniam,
quis nostrud exercitation ullamco laboris nisi ut aliquip ex ea commodo
consequat. Duis aute irure dolor in reprehenderit in voluptate velit esse
cillum dolore eu fugiat nulla pariatur. Excepteur sint occaecat cupidatat non
proident, sunt in culpa qui officia deserunt mollit anim id est laborum.








\subsection{Формула Ландауэра}

Рассмотрим два куска металла между которыми существует 1D идеальный провод. Химпотенциалы соответственно равны
\begin{equation*}
	\mu_L = \mu + \frac{1}{2} eV, 
	\hspace{5 mm} 
	\mu_R = \mu - \frac{1}{2} eV,
\end{equation*}
ток можем найти как $I = T_R-I_L$:
\begin{equation*}
	I = \sum_{k > 0} e v_k \left(
		f_l(\varepsilon_k) - f_k (\varepsilon_k)
	\right),
\end{equation*}
где $f_{R, L} (\varepsilon_k) = f(\varepsilon_k - \mu_{R, L})$ -- числа заполнения. 
Подставляя в выражение для тока, находим
\begin{equation*}
	I = -\sum_{k > 0} e v_k \frac{\partial f}{\partial \varepsilon_k} \left(
		\delta \mu_L - \delta \mu_R
	\right) = - e^2 V \int_{k>0} \frac{d k}{2\pi} \frac{\partial \varepsilon_k}{\partial \hbar k}  \frac{\partial f}{\partial \varepsilon_k} = - \frac{e^2}{2\pi} V \int_{0}^{\infty} \d \varepsilon \frac{\partial f}{\partial \varepsilon} =  \frac{e^2}{2\pi} V.
\end{equation*}
где сделали подстановку $v_k = {\partial \varepsilon_k}/{\partial \hbar k}$, числа заполнения равны $f(\varepsilon=0)=1$ в $f(\varepsilon=\infty)= 0$ соответственно. Таким образом находим квант проводимости
\begin{equation*}
	G = \frac{I}{V} = \frac{e^2}{2 \pi \hbar}.
\end{equation*}
Если скажем, что электроны отражаются с коэффициентом $|t_i|^2$ и всего всего есть $N$ одномерных каналов, получим \textit{формулу Ландауэра}
\begin{equation*}
	G = \frac{e^2}{2\pi \hbar} \sum_{i=1}^N |t_i|^2.
\end{equation*}





\subsection{Квантовый эффект Холла}


\textbf{Эффект Холла}. 
Вспоминим уравнение Друде
\begin{equation*}
	m \left( \cancel{\frac{d \vc{v}}{d t}}  + \frac{1}{\tau} \vc{v}\right) = e \vc{E} + \frac{e}{c} \vc{v} \times  \vc{B},
\end{equation*}
где рассматриваем 2D образец, смотрим стационарное решение. Расписывая покомпоненто, можем найти
\begin{equation*}
	\begin{pmatrix}
	    1 & -\frac{e B \tau}{mc}  \\
	    \frac{e B \tau}{mc} & 1  \\
	\end{pmatrix} \begin{pmatrix}
		v_x \\ v_y
	\end{pmatrix} = \frac{e \tau}{m} \begin{pmatrix}
		E_x \\ E_y
	\end{pmatrix}.
\end{equation*}
Подставим $\vc{J} = n e \vc{v}$, тогда
\begin{equation*}
	\begin{pmatrix}
	    1 & -\frac{e B \tau}{mc}  \\
	    \frac{e B \tau}{mc} & 1  \\
	\end{pmatrix} 
	\begin{pmatrix}
		J_x \\ J_y
	\end{pmatrix} = \frac{n e^2 \tau}{m} 
	\begin{pmatrix}
		E_x \\ E_y
	\end{pmatrix}.
\end{equation*}
Получаем соотношение $\vc{J} = \hat{\sigma} \vc{E}$, тогда $\vc{E} = \hat{\sigma}^{-1} J = \hat{\rho} \vc{J}$. Матрица $\hat{\rho}$ получается равной 
\begin{equation*}
	\hat{\rho} = \frac{m}{n e^2 \tau} \begin{pmatrix}
	    1 & -\frac{e B \tau}{mc}  \\
	    \frac{e B \tau}{mc} & 1  \\
	\end{pmatrix}.
\end{equation*}
Итого находим
\begin{equation*}
	\rho_{xx} = \frac{m}{n e^2 \tau},
	\hspace{10 mm} 
	\rho_{xy} = \frac{B}{n |e| c}.
\end{equation*}




\begin{wrapfigure}{r}{0.2\textwidth}
  \begin{center}
        \vspace{-7 mm}
        \includegraphics[width=0.9\linewidth]{figures/Ресурс 2.pdf}
  \end{center}
  \vspace{-5mm}
    \caption{Уровни Ландау}
    %\label{fig:}
\end{wrapfigure}



\textbf{Квантовый эффект Холла}. Характерная длина основного состояния
\begin{equation*}
	\lambda_B = \sqrt{\frac{\hbar}{m \omega_B}},
	\hspace{10 mm} 
	\omega_B = \frac{e B}{mc}.
\end{equation*}
Подставляя $\omega_B$, находим
\begin{equation*}
	\lambda_B = \sqrt{\frac{\hbar c}{e B}},
\end{equation*}
где $[\hbar c/e]$ -- квант потока $B$. 



Далее рассматриваем образцы такие, чтобы их характерный размер $L \gg \lambda_B$. Уровни Ландау загибаются к краям образца. Далеко от границ все электронные состояния являются локализованными и не участвуют в процессах электропередачи. Неупругие процессы отсутсвуют в сиду заполненности сферы Ферми. 




По каналам проводимости вдоль границ будет одномерный проводник с квантущимся conductance $G$:
\begin{equation*}
	G = \frac{e^2}{h},
	\hspace{10 mm} 
	\sigma_{xy} = \frac{e^2}{h} \nu,
\end{equation*}
где $\nu$ -- количество заполненных уровней Ландау. Для $\sigma_{xx} = 0$ в силу локализованности электронов. Тогда для сопротивления $\hat{\sigma} = \hat{\rho}^{-1}$, получаем
\begin{equation*}
	\sigma_{xx} = \frac{\rho_{xx}}{\rho_{xx}^2 + \rho_{xy}^2},
	\hspace{5 mm} 
	\sigma_{xy} = \frac{-\rho_{xy}}{\rho_{xx}^2 + \rho_{xy}^2}.
\end{equation*}
Если $\sigma_{xx} = 0$, тогда $\rho_{xx} = 0$ и $\rho_{xy} = 1/\sigma_{xy}$:
\begin{equation*}
	\rho_{xy} = \frac{h}{e^2} \frac{1}{\nu}.
\end{equation*}



Емкость уровней Ландау 
\begin{equation*}
	N_{LL} = \frac{\Phi}{\Phi_0} = \frac{S}{2 \pi \lambda_B^2},
	\hspace{0.5cm} \Rightarrow \hspace{0.5cm}
	\nu = \frac{N}{N_{LL}} = \frac{N \lambda_B^2 2\pi}{S} = 2 \pi n \lambda_B^2 = 2 \pi n \frac{\hbar c}{e B},
	\hspace{0.5cm} \Rightarrow \hspace{0.5cm}
	\frac{1}{\nu} \sim B.
\end{equation*}
Идеальности будет мешать эффект туннелирования, но они подавлены как $e^{-L/\lambda_B}$.





