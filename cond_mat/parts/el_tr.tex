% Электронный транспорт в металлах
% Классическая модель проводимости Друде.
% Полуклассическое описание электронного транспорта - уравнение Больцмана.
% Проводимость при упругом рассеянии электронов на примесях, транспортное время рассеяния.


\subsection{Модель проводимости Друде}

Рассмотрим движение электронов под действием электрического поля
\begin{equation*}
	m (\ddot{x} + \gamma \dot{x}) = e E,
\end{equation*}
в установившемся режиме $\ddot{x} = 0$, $\gamma = 1/\tau$, где $\tau$ -- время столкновений. Так находим
\begin{equation*}
	v = \frac{e \tau}{m}E,
	\hspace{5 mm} 
	j = e n v = \frac{n e^2}{m} \tau E,
	\hspace{0.5cm} \Rightarrow \hspace{0.5cm}
	\sigma_{\scalebox{0.55}{\text{D}}} = \frac{n e^2}{m} \tau.
\end{equation*}

Столкновения обусловлены нарушением структуры решетки, электрон-электронным взаимодействием, рассеянием на фононах. Время $\tau$ при разных механизмах можем найти по формуле
\begin{equation*}
	\frac{1}{\tau} = \frac{1}{\sub{\tau}{прим}} + \frac{1}{\sub{\tau}{фонон}} + \ldots
\end{equation*}



\subsection{Формула Больцмана}

Запишем уравнение на динамику чисел заполнения $f(k, r)$:
\begin{equation*}
	\partial_t f(k, r) + \vc{v} \cdot \vc{\nabla}_r f + \dot{\vc{k}} \vc{\nabla}_k f = I_{st}[f],
\end{equation*}
где $I_{st}[f]$ -- интеграл столкновений, $\dot{\vc{k}} = eE / \hbar$ -- внешняя сила. Считая ситуацию изотропной и стационарной, получаем \textit{уравнение Больцмана}
\begin{equation*}
	e \vc{E} \cdot \frac{\partial f}{\partial \vc{k}} = I_{st}[f].
\end{equation*}
Столкновительный интеграл можем расписать в виде
\begin{equation*}
	I_{st}[f] = - \sum_{k'} \left(
		w_{k'k} f_k (1-f_{k'}) - w_{kk'} f_{k'} (1-f_k)
	\right) = \sum w_{k k'} (f_k - f_{k'}),
\end{equation*}
где амплитуда рассеяния по золотому правилу Ферми с учетом упругости процесса $w_{k'k} = \frac{2 \pi}{\hbar} |V_{k' k}|^2 \delta(\varepsilon_k - \varepsilon_{k'})$. Можем угадать решение в виде
\begin{equation*}
	f_k = f(\varepsilon_k) + \delta f_k,
	\hspace{5 mm} 
	\delta f_k = - c \cdot \vc{E} \cdot \vc{v}_k \cdot f_0'(\varepsilon).
\end{equation*}
Подставляя, найдём связь тока и поля в виде
\begin{equation*}
	\vc{J} = \frac{n e^2}{m} \tau \vc{E},
	\hspace{5 mm} 
	\frac{1}{\tau} = -\lambda = \int \frac{\d \Omega}{4 \pi} w(\theta) (1-\cos \theta).
\end{equation*}
Таким образом вклад вносят только рассеивающие на большой угол примеси. 



\textbf{Критерий применимости}. Здесь приближение было в разделение координат и импульсов, то есть должно быть $\Delta k \sim \frac{1}{l} \ll \kF$, где $l \sim \vF \cdot \tau$, так получаем критерий \textit{Иоффе-Регеля}
\begin{equation*}
	\kF \cdot l \gg 1
\end{equation*}
критерий применимости Больцмановского приближения. 
