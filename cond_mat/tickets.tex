% Вопросы к экзамену по курсу 
% «Введение в физику конденсированных сред»
% А. Н. Рубцов, В.И. Юдсон
% декабрь 2021
	 
	Основные понятия и приближения 
	Атомные масштабы и малые параметры в теории конденсированных сред.
	Частицы и квазичастицы. Модель желе. Метод функционала плотности.

Электроны в периодическом потенциале 
Теорема Блоха. Зонная структура изоляторов и металлов.
Простые решеточные модели. Дисперсия электронов в одномерных цепочках.
 
Электронные возбуждения в материалах с запрещенной зоной: экситоны
Экситоны Ванье - Мотта в полупроводниках,
Экситоны Френкеля в молекулярных кристаллах.

Фононы
Акустические и оптические фононы (пример одномерной цепочки).
Модель Дебая.

	Эффекты взаимодействия электронов в металле
	Концепция Ландау Ферми-жидкости.
	Кулоновское взаимодействие в металлах, экранирование внешнего потенциала (приближение Томаса-Ферми)
	Диэлектрическая функция в металлах. Плазмоны.

	Электронный транспорт в металлах
	Классическая модель проводимости Друде.
	Полуклассическое описание электронного транспорта - уравнение Больцмана.
	Проводимость при упругом рассеянии электронов на примесях, транспортное время рассеяния.

	Квантовые эффекты в проводниках
	Ограничения полуклассического описания электронного транспорта уравнением Больцмана. Критерий Иоффе-Регеля.
	Локализация Андерсона в сильном беспорядке.
	Формула Ландауера для кондактанса.
	Введение в квантовый эффект Холла.  

	Фазовые переходы
	Параметр порядка и теория Ландау фазовых переходов. 
	Флуктуации. Голдстоуновские моды.  
	Критерий Гинзбурга-Леванюка, верхняя критическая размерность.
	Фазовый переход в цепочке Изинга в поперечном поле.

	Магнетизм
	Природа магнетизма. Локализованные магнитные моменты. 
	Обменное взаимодействие. 
	Критерий Стонера для магнетизма делокализованных электронов. 
 
Макроскопические квантовые явления
Введение в сверхтекучесть. Критерий сверхтекучести Ландау. Бозе-эйнштейновская конденсация в идеальном бозе-газе. Теория сверхтекучести Боголюбова для слабо-неидеального газа.