\subsection{Неравенство Рао-Крамера}


Введем множество $A = \{x \mid f(x;\, \vc{\theta}) \neq 0\}$. \textit{Первое условие регулярности} -- $A \neq A(\vc{\theta})$. 
\textit{Вторым условием регулярности} является дифференцируемость $f(x;\, \theta)$ по $\vc{\theta}$ на $A$. 
Наконец, введем $u = \partial_\theta \ln f(\xi;\, \vc{\theta})$, и условием регулярности будет выступать
\begin{equation*}
	\E u = 0 \ \ \forall \theta \in \Theta,
	\hspace{10 mm} 
	0 < \D u < \infty.
\end{equation*}

К слову про дипсерсию
\begin{equation*}
	\D u = \E \left(\frac{\partial \ln f}{\partial \theta} \right)^2 \overset{\mathrm{def}}{=} I_1 (\theta),
\end{equation*}
где $I_1 (\theta)$ -- информация Фишера, или $I_n (\theta) = n I_1(\theta)$:
\begin{equation*}
	I_n (\theta) \overset{\mathrm{def}}{=}  \E \left(\partial_\theta \ln L(\xi_1\seq \xi_n; \theta)\right)^2.
\end{equation*}

\begin{to_thr}[неравенство Рао-Крамера]
    Пусть выполнены условия регулярности, $\hat{\theta}_n$ -- несмещенная оценка $\theta$ и $\E \hat{\theta}_n$ можно дифференцировать по $\theta$ под знаком интеграла. Тогда\footnote{
    	Где $I_n (\theta) \sim n \cdot \const$.
    }  $\D \hat{\theta}_n \geq \frac{1}{I_n (\theta)}$. 
\end{to_thr}

\begin{to_thr}[]
    Пусть выполнены условия регулярности, $f(x; \vc{\theta})$ трижды дифференцируема по $\theta$, $\E \partial_\theta^3 f(\xi; \theta)\leq h(\xi)$ с $\E h < \infty$. Тогда решение уравнения $\partial_\theta \ln L = 0$ является ОМП.  Эта оценка асимптотически нормальна с наименьшей дисперсией $\frac{1}{I_1(\theta)}$ -- асимптотически эффективна. 
\end{to_thr}



% \section{Доверительные интервалы}


% \red{дописать}