Рассмотрим выборку $x_1 \seq x_n$. Предположим, что есть набор
\begin{equation*}
	\xi_1 \seq \xi_n \colon  \Omega \mapsto \mathbb{R},
	\hspace{5 mm} 
	\xi_1(\omega) = x_1 \seq \xi_n(\omega) = x_n.
\end{equation*}
Верим, что $\xi_i$ \textit{независимы в совокупности} и одинаково распределены. 

% разберемся с тремя классами задач

Предполагаем, что $F_{\xi_i} \in \{F_{\vc{\theta}}\}_{\vc{\theta} \in \Theta \subset \mathbb{R}^m}$, где $\Theta$ -- параметрическое множество.

\begin{to_def}
    Оценка (\textit{точечная}) -- любая функция $\hat{\theta}_n$ элементов выборки $\colon  \Im \hat{\theta}_n \in \Theta$. 
\end{to_def}

Вообще оценка это $\hat{\theta}_n (\xi_1 \seq \xi_n)$ -- случайная величина. Естественно определить \textit{несмещенную} оценку 
\begin{equation*}
	\E \hat{\theta}_n = \theta
\end{equation*}
и \textit{состоятельную} оценку
\begin{equation*}
	\hat{\theta}_n \toP \theta,
\end{equation*}
где имеется ввиду \textit{сходимость по вероятности}:
\begin{equation*}
	\forall \varepsilon > 0 \ P(|\hat{\theta}_n -\theta| > \varepsilon) \ton 0.
\end{equation*}

Рассмотрим, например, нормальное распределение $\xi_i \sim \mathcal N(\theta_1, \theta_2^2)$. Положим
$\hat{\theta}_n = \bar{x} = \frac{1}{n} \sum_{i=1}^{n} \xi_i$, тогда
\begin{equation*}
	\E \hat{\theta}_n = \frac{1}{n} \sum_{i=1}^{n} \E \xi_i = \theta_1,
\end{equation*}
то есть оценка несмещенная. Более того, оценка состоятельна
\begin{equation*}
	\hat{\theta}_n = \frac{1}{n}\left(\xi_1 + \ldots + \xi_n\right) \toP \E \xi_i = \theta_1,
\end{equation*}
из закона больших чисел\footnote{
	Может быть усилено до сходимость почти наверное. Тогда оценка будет называться сильно состоятельной.
}.

Теперь для дисперсии в $\xi_i \sim \mathcal N(a, \theta_2^2)$ разумно взять
\begin{equation*}
	\hat{\theta}_n = \frac{1}{n} \sum_{i=1}^{n} (\xi_i - a)^2,
\end{equation*}
которая является состоятельной и несмещенной.

Но, для неизвестного среднего $\xi_i \sim \mathcal N(\theta_1, \theta_2^2)$ уже по-другому
% https://youtu.be/aJokwg6c2KQ?list=PLthfp5exSWErTVWq4cVtRXDw5MqBqavJ1&t=3120
\begin{equation*}
	\hat{\theta}_n = \frac{1}{n} \sum_{i=1}^{n} \left(
		\xi_i - \frac{1}{n} \sum_{j=1}^{n} \xi_j
	\right)^2,
	\hspace{5 mm} 
	\E \hat{\theta}_n = \frac{n-1}{n} \theta_2^2,
\end{equation*}
получили смещенную оценку (асимптотически несмещенную), а для несмещенной оценки необходимо брать множитель $\frac{1}{n-1}$.

