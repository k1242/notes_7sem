
\subsection{Построение статистических оценок: метод моментов}

Понятно, что можем построить состоятельные несмещенные оценки, вида
\begin{equation*}
	\E \xi_i^k \ \colon  \ \overline{x^k} = \frac{1}{n} \sum_{i=1}^{n} \xi_i^k.
\end{equation*}
Зная, что величины распределены по $F_{\xi_i} \in \{F_{\vc{\theta}}\}_{\vc{\theta} \in \Theta \subset \mathbb{R}^m}$, можем посчитать
\begin{equation*}
	\E \xi_i^k = f_k(\theta_1 \seq \theta_m).
\end{equation*}
Составим систему уравнений 
\begin{equation*}
	f_1 (\theta_1 \seq \theta_m) = \bar{x}, \hspace{5 mm} \ldots, \hspace{5 mm} f_m (\theta_1 \seq \theta_m) = \overline{x^m},
\end{equation*}
которую будем считать разрешимой и с $\exists !$ решением -- $\hat{\theta}_n^1 \seq \hat{\theta}_n^m$, искомой оценкой параметров распределения. 




\subsection{Построеник статистических оценок: метод максимального правдоподобия}

Рассмотрим $f(x;\, \vc{\theta}) = P_\theta(\xi=x)$ для дискретного случая и $f(x;\, \vc{\theta}) = p_\xi (x;\, \vc{\theta})$ для абсолютно непрерывного случая. 
Определим \textit{функцию правдоподобия} 
\begin{equation*}
	L(x_1\seq x_n; \vc{\theta}) = \prod_{i=1}^n f(x_i;\, \vc{\theta}),
\end{equation*}
то есть чем вероятнее для $\vc{\theta}$ наблюдение таких параметров, тем больше функция правдоподобия. Оценка метода правдоподобия (ОМП) -- это точка $\hat{\theta}_n$ максимума функции $L$. 

Удобно работать с $\ln L$, тогда будем решать систему
\begin{equation*}
	\frac{\partial \ln L}{\partial \theta_1} = 0, \hspace{5 mm} \ldots, \hspace{5 mm} \frac{\partial \ln L}{\partial \theta_m} = 0,
\end{equation*}
и находить ОМП.


Например, для Binom$(k, \theta)$ 
\begin{equation*}
	P(\xi_i = x_i) = C_k^{x_i} \theta^{x_i} (1-\theta)^{k-x_i},
	\hspace{5 mm} 
	L(x_1 \seq x_n; \theta) = \prod_{i=1}^{n} C_k^{x_i} \theta^{x_i} (1-\theta)^{k-x_i},
\end{equation*}
тогда можем найти
\begin{equation*}
	\frac{\partial \ln L}{\partial \theta} = \sum_{i=1}^{n} \left(
		\frac{x_i}{\theta} - \frac{k-x_i}{1-\theta}
	\right) = 0,
	\hspace{0.5cm} \Rightarrow \hspace{0.5cm}
	\hat{\theta}_n = \frac{1}{kn} \sum_{i=1}^{n} x_i.
\end{equation*}

